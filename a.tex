% !TEX program = xelatex
\documentclass{ecustbachelorthesis}
% \graphicspath{{img/}}
% \renewcommand{\thetable}{\arabic{table}}
% \renewcommand{\thefigure}{\arabic{figure}}
\input{_}
\renewcommand{\thesistype}{}
\renewcommand{\thesistitle}{\realthesistitle}
\updatecmd
\usepackage{datetime}
\hypersetup{
  pdfinfo={
    Author={\realauthorname},
    Title={\thesistitle{}\thesistype},
    CreationDate={D:20150315102933},
    ModDate={D:\pdfdate},
    Keywords={\realkeywords},
    Subject={MOOC数据结构课程动态演示系统}
  }
}

% \usepackage{showframe}
\begin{document}
% \label{title:t1}
% \pdfbookmark[0]{标题}{title:t1}
% !TEX root = ../a.tex
\begin{abstractzh}{MOOC,数据结构课程,动态演示}
MOOC指的是大规模开放在线课堂。随着计算机水平的迅速发展,MOOC课程中的数据结构课程吸引了世界各地的学习者。课程包括数据结构和算法,其概念都比较抽象,对初学者来说学习难度较大。基于MOOC的数据结构课程动态演示系统发挥MOOC课程教学的特点,帮助学习者理解其概念。研究主要通过Javascript技术,设计并实现一个可以嵌入到MOOC课程中的动态演示系统,其中演示内容包括基本的数据结构及其应用和算法,系统结合动画和教学语言代码,展示数据结构课程中基本的抽象概念和具体操作。
\end{abstractzh}

\begin{abstracten}{MOOC, Data Structure, animation}
Content content content content content content content content content content content content content content content content content content content content content content content content content content content content content
\end{abstracten}

\mktableofcontents
% % !TEX root = ../o.tex
\chapter{研究背景}
% \ifthenelse{\equal{\thesistype}{}}{正文}{不是正文}

\section{背景介绍}
\begin{sectext}
MOOC(Massive Open Online Course)指的是大规模开放在现课堂。2012年起,许多知名大学陆续登录到在线学习网络平台,在网上提供免费课程。现在,著名的在网络平台包括Coursera、Udacity和edX等。随着课程提供商的兴起,世界各地的学生拥有了更多系统学习的机会。这些平台和真正的大学一样,提供的课程大部分针对高等教育,有一套自己的学习和管理系统。

由于计算机和互联网技术的迅速发展,计算机相关的课程在MOOC上也都变得非常热门,其中也包括数据结构这门课程。数据结构课程作为计算机科学中一门综合性较强的专业基础课程,研究的内容包括非数值计算程序设计问题中计算机操作的对象以及它们之间的关系和运算等。而这部分内容往往都是抽象的,初学者通常难以对课程中的概念有比较直观的理解。

本研究通过结合MOOC课程所具有的特征,设计并实现一个数据结构课程动态演示系统。
\end{sectext}
\section{研究意义}
\begin{sectext}
数据结构课程动态演示能够帮助学习者更好地理解抽象的数据结构概念和复杂的算法步骤。系统通过把抽象的数据结构复杂的算法转换为形象生动的动画过程,清晰地向学习者展现数据结构的操作步骤,在演示动画中可以关注每一个执行细节,这样更有利于课后自己动手编写数据结构和算法相关的程序。同时,学习者也不会对于数据结构或算法的概念亦或是编程感到枯燥无聊,更加积极得参与课程的学习中。

本研究基于MOOC的特点对数据结构课程动态演示系统进行设计并实现,进一步补充了MOOC课程中所存在的不足之处。MOOC课程是将传统课程讲授与在线网络相结合,一方面发挥了网络优势,另一方面也带来了平台自身的局限性,其特点集中表现在五个方面,具体如下:
\begin{itemlist}
\item 大多数的在线课程均有自己的上课周期,大约为1--3个月。
\item 上课内容碎片化,主要以知识点进行划分。
\item 多数课程为教师引导型授课。因为缺乏师生互动性,要想到达理想授课效果,需要把讲授知识简化、串引,并附有生动易于理解的动画。
\item 授课除包含教师讲解外,还会有相应的作业和考试机制。每门课程都有相应的教师与学生的讨论.教师需要在授课以外时间完成学生的各种提问。
\item 与传统课堂教学相比MOOC课程的学生来自全球各地语言的互通成为了MOOC课程翻译的重要一步。
\end{itemlist}

针对以上五点,数据结构课程动态演示系统的设计与实现将帮助参与到MOOC数据结构课程的学习者更有效的学习。与以往在课程的课件中添加参考资料或者外部访问的链接不同,将系统嵌入到MOOC的课程中可达到以下三个效果:
\begin{itemlist}
\item 知识点集中,分类目录清晰,帮助学习者全面掌握并巩固课程基本概念。
\item 交互式动态演示和图形化界面可以提高学习者的兴趣,消除学习者的语言障碍,从而更有效地参与到在线的课堂和课后的讨论中。
\item 对于基本的习题答疑,可以直接参考动态演示内容,有助于提高学习效率。
\end{itemlist}
\end{sectext}

% % !TEX root = ../../t.tex
\chapter{现有的算法可视化网站}
\begin{sectext}
本节总结了我们对于截至2012年4月不同算法可视化网站的分析。以下的报告不考虑非在线的算法可视化。所有的算法的来自与``高级编程2''这本手册(Halim和Halim,2011)。在分析中,我们发现经典算法和非经典算法数量上的不平衡。经典算法——通常比较简单——例如排序、有序数列的二分查找等,而非经典和少见的算法——通常比较困难——例如树状数组(Fenwick,1994)、有向图的强连通分支(Tarjan,1972),网络的最大流(Edmonds和Karp,1972)等。
\end{sectext}
\section{单一可视化}
\begin{sectext}
我们把得到的记过分为两类:单一的可视化和统一可视化。表\ref{tab:1}中为单一的现有算法可视化网站,这些网站只有一到三个算法动画,开发者没有可视化其他的算法。这增加了学生的学习的负担,学生们在学习不同算法时不得不访问不同的网站。为了简化表\ref{tab:1},我们只选择了几个有代表性的经典算法可视化的网站。
% !TEX root = ../../t.tex
\begin{center}
\setstretch{1.35}
\vspace{-20pt}
\begin{longtable}{p{4cm} p{8cm} l l l}
\caption{单一可视化(截至2012年4月所有URL地址均可以访问)}
\label{tab:1} \\

\hline \multicolumn{1}{l}{主题} & \multicolumn{1}{l}{算法可视化网站地址} & \multicolumn{1}{c}{N} & \multicolumn{1}{c}{H} & \multicolumn{1}{c}{O}\\ \hline
\endfirsthead

\multicolumn{5}{r}%
{接表 \ref{tab:1}} \\
\hline \multicolumn{1}{l}{主题} & \multicolumn{1}{l}{算法可视化网站地址} & \multicolumn{1}{c}{N} & \multicolumn{1}{c}{H} & \multicolumn{1}{c}{O}\\ \hline
\endhead

\hline \multicolumn{5}{r}{待续}
\endfoot

\hline
\endlastfoot

{各种线性数据结构} & \url{apbrwww5.apsu.edu/myersb2/linked_list_animation.htm} & & H & O \\
& \url{www.u-www.cosc.canterbury.ac.nz/mukundan/dsal/StackAppl.html} & & H & O \\
{各种排序算法} & \url{www.cs.auckland.ac.nz/∼jmor159/PLDS210/Java/q_sort/tqs_new.html} & & H & O \\
& \url{www.cse.iitk.ac.in/users/dsrkg/cs210/applets/sortingII/mergeSort/mergeSort.html} & & & O \\
& \url{www.cs.oswego.edu/∼mohammad/classes/csc241/samples/sort/Sort2-E.html} & & H & O \\
位掩码或位操作 & (见4.1) & N & & \\
{基本的二叉搜索树(表格)} & \url{aleph0.clarku.edu/~achou/cs102/examples/bst_animation/BST-Example.html} & & & O \\
& \url{www.cs.jhu.edu/~goodrich/dsa/trees/btree.html} & & & O \\
& \url{www.csc.liv.ac.uk/~ullrich/COMP102/applets/bstree/} & & & O \\
{平衡树(AVL)} & \url{webdiis.unizar.es/asignaturas/EDA/AVLTree/avltree.html} & & & O \\
& \url{www.site.uottawa.ca/~stan/csi2514/applets/avl/BT.html} & & & O \\
& \url{www.cs.jhu.edu/~goodrich/dsa/trees/avltree.html} & & & O \\
{堆(优先队列)} & \url{nova.umuc.edu/~jarc/idsv/lesson2.html} & & & O \\
& \url{www.cs.auckland.ac.nz/~jmor159/PLDS210/heaps.html} & & & O \\
{图数据结构(邻接矩阵或邻接表)} & (\url{nova.umuc.edu/~jarc/idsv/} 中有一部分相关主题) & N & & \\
& (见4.4) & & & \\
{并查集} & \url{research.cs.vt.edu/AVresearch/UF/} & & & O \\
& \url{www.cs.unm.edu/~rlpm/499/uf.html} & & & O \\
Segment tree & (我们的项目今后会增加这种算法) & N & & \\
树状数组 & (见4.3) & N & & \\
{N皇后问题} & \url{www.animatedrecursion.com/advanced/the_eight_queens_problem.html} & & H & O \\
& \url{yuval.bar-or.org/index.php?item=9} & & H & O \\
经典动态规划 & (今后会增加这种算法) & N & & \\
{图的遍历:BFS/DFS等} & \url{www.cs.sunysb.edu/~skiena/combinatorica/animations/search.html} & & H & \\
& \url{www.rci.rutgers.edu/~cfs/472_html/AI_SEARCH/SearchAnimations.html} & & H & \\
割点、桥、强连通分量 & (见4.4) & N & & \\
{MST (Prim算法)} & \url{www.unf.edu/~wkloster/foundations/PrimApplet/PrimApplet.htm} & & H & O \\
& \url{students.ceid.upatras.gr/~papagel/project/prim.htm} & & H & O \\
{MST (Kruskal算法)} & \url{www.unf.edu/~wkloster/foundations/KruskalApplet/KruskalApplet.htm} & & H & O \\
& \url{students.ceid.upatras.gr/~papagel/project/kruskal.htm} & & H & O \\
SSSP (Bellman Ford算法)& (见4.7) & N & & \\
{SSSP (Dijkstra算法)} & \url{www.unf.edu/~wkloster/foundations/DijkstraApplet/DijkstraApplet.htm} & & H & O \\
& \url{www.dgp.toronto.edu/~jstewart/270/9798s/Laffra/DijkstraApplet.html} & & & O \\
APSP (弗洛伊德算法) & \url{www.pms.ifi.lmu.de/lehre/compgeometry/Gosper/shortest_path/shortest_path.html} & & H & O \\
网络流(Edmonds Karp算法) & \url{felix-halim.net/research/maxflow/index.php} & & H & O \\
&\url{www.eecs.wsu.edu/∼cook/aa/lectures/applets/ek/MaxFlowHome.html} & & H & O \\
DAG算法 & (今后会增加这种算法,包括:拓扑排序、最短/长路径、路径数) & N & & \\
树上的算法 & (今后会增加这种算法,包括:最短/长路径、所有点对最短路径、最近公共祖先) & N & & \\
二分图算法 & \url{www-b2.is.tokushima-u.ac.jp/~ikeda/suuri/main/index.shtml} & & H & O \\
字符串匹配 & \url{cgjennings.ca/fjs/index.html} & & & O \\
{后缀树} & \url{illya-keeplearning.blogspot.com/2009/06/suffix-trees-java-applet.html} & & & O \\
后缀数组 & \url{felix-halim.net/pg/suffix-array} & & & \\
多边形算法 & (见4.9) & N & & \\
% \multirow{3}{*}{\begin{minipage}{4cm}二维凸包 (Graham扫描算法,Andrew单调链算法,Jarvis卷包裹算法)\end{minipage}} & \url{www.cs.princeton.edu/courses/archive/spr10/cos226/demo/ah/GrahamScan.html} & & H & O \\
% \multirow{3}{*}{\parbox{4cm}{二维凸包 (Graham扫描算法,Andrew单调链算法,Jarvis卷包裹算法)}} & \url{www.cs.princeton.edu/courses/archive/spr10/cos226/demo/ah/GrahamScan.html} & & H & O \\
凸包 (Graham扫描、Andrew单调链、Jarvis卷包裹算法) & \url{www.cs.princeton.edu/courses/archive/spr10/cos226/demo/ah/GrahamScan.html} & & H & O \\
&\url{nms.lcs.mit.edu/ ∼ aklmiu/6.838/convexhull/} & & & O \\
&\url{www.cs.unc.edu/∼snoeyink/demos/ch/Jarvis.html} & & & O \\
\end{longtable}
\end{center}

\vspace{-5pt}

在同一个表中,我们用以下记号强调了三个可能的问题,这些问题源自我们的目标,已经在第一节中列出:
\begin{itemlist}
\item N(少见):这种算法少见,在因特网上用相关关键词搜索后没能得到可视化网站的结果。

\item H(静态,非交互式):这种算法使用内嵌的例子(包括使用动态的GIF图),用户不能通过输入新的数据来观察算法如何执行。

\item O(技术过时):这种(过时的)算法可视化使用的是以前的技术(例如Flash、Java小应用程序),不完全支持便携式设备,通过现代的智能手机(例如iPhone)或者平板电脑(例如iPad)不能访问。

\end{itemlist}

\end{sectext}
\section{统一可视化}
\begin{sectext}
除了表\ref{tab:1}中的单一可视化,我们也发现以下的统一可视化,这些可视化网站提供了三个以上的可视化:
\begin{itemlist}
\item  \url{algoviz.org} 基本上作为一个网站入口,题够不同算法可视化的网站链接。以后我们的可视化网站也将列入其中。但是,截至2012年4月,一些第三节中列出的算法可视化(例如树状数组、强连通分量等)还没有被列出。

\item  \url{www.ansatt.hig.no/frodeh/algmet/animate.html} 是另外一个网站入口,同样手机了其他可视化网站的链接。(与 \url{algoviz.org} 本质相同)

\item  \url{nova.umuc.edu/~jarc/idsv/} 提供的可视化包括:二叉树、几种图的表示和算法以及排序。

\item  \url{www.csse.monash.edu.au/~dwa/Animations/index.html} 提供的一些算法可视化与 \url{nova.umuc.edu/~jarc/idsv/} 类似。

\item  \url{research.cs.vt.edu/AVresearch/} 提供一部分算法可视化,从2003年到2009年进行维护开发。

\item  \url{www.cs.usfca.edu/~galles/visualization/Algorithms.html} 是最接近统一可视化的网站,与本论文提出的观点最为相近。同时该网站使用HTML5技术。我们会在第三节中突出我们的可视化网站与该网站的不同之处。
\end{itemlist}
\end{sectext}

% % !TEX root = ../o.tex
% \vspace*{-13pt}
\chapter{技术路线}
\begin{chatext}
研究将设计并实现一个基于MOOC的数据结构动态演示系统,采用web技术使得该系统可以嵌入到MOOC课程网站。系统使用的主要技术为Javascript技术,研究的主要内容为数据结构和动态演示。
\end{chatext}
\section{Javascript}
\begin{sectext}
Javascript是一种属于网络的脚本语言,已经被广泛用于Web应用开发,常用来为网页添加各式各样交互行为,具有良好的跨平台性,在大多数浏览器的支持下,能够完成许多出色的功能。

在基于MOOC的数据结构动态演示系统中,Javascript主要用来创建数据结构的动态演示,提高网页的交互性。其主要的开发平台有浏览器控制台和node.js平台。浏览器控制台可用于设计、实现过程中的调试,将最终系统的展示在浏览器页面中;而node.js开发平台是一个基于Chrome Javascript运行时平台,有自己的包管理工具,用于方便、快捷的搭建易于扩展的网页应用。
\end{sectext}
\section{数据结构和动态演示}
\begin{sectext}
系统中涉及的数据结构来源于MOOC数据结构课程的教学大纲,同时也参考传统教学的教材中所涉及到的基本数据结构和算法。系统中演示数据结构的参考源代码采用相应教学语言(C语言)进行编写,这些代码与系统中的动画同步演示,可以帮助学习者加深对代码的理解,在实际编程中更好地应用数据结构。演示的内容主要由数据结构和算法构成,其中数据结构不仅仅是数据结构本身的概念,而是通过演示其在算法或实际应用中的初始化、具体操作步骤等以达到课程更加生动有趣的效果。

系统中演示的数据结构及其应用如下:
\begin{itemlist}
\item 栈:算术表达式。
\item 队列:杨辉三角。
\item 矩阵:稀疏矩阵的存储压缩。
\item 树:哈夫曼编码、平衡二叉树。
\item 图:拓扑排序、强连通分量、Dijkstra最短路、Prim最小生成树。
\item 散列表:开地址、链地址解决冲突。
\end{itemlist}

系统中演示的算法如下:
\begin{itemlist}
\item 排序:归并排序、快速排序。
\item 查找:顺序查找、二分查找。
\end{itemlist}

动态演示的实现包括实现其界面绘制、图形绘制和事件的处理,这些则全部由Javascript技术支持。

首先,演示界面相对统一,对多个不同内核的浏览器进行兼容。既可以通过编写简单的HTML和CSS实现,也可以加入相应的Javascript UI库对网页调整。

其次,数据结构图形对于不同的数据结构会有所差异,但一般都是由基本图形构成,即线段、弧线、矩形、圆形等,这些在HTML页面上的绘制可以通过D3.js库完成。D3.js是最流行的可视化库之一,基于数据对HTML文档树进行操作,常用于将数据转化为交互性的图形,支持SVG和Canvas两种模式。

最后,事件的处理,也就是页面交互以及动画过程,需要结合Javascript的设计模式。发布者—订阅者模式,或者是观察者模式,都是处理事件常用的模式。这种模式定义了一种一对多的依赖关系,可以有效的解决对象互相依赖过程中产生的副作用,维护相关对象的一致性。

以上三个模块,由相互独立的Javascript库完成,借助node.js的模块化开发的特点,将模块整合成一个系统的模块,单元间可以共享资源,例如:演示界面的元素等。而每个系统的模块作为一种数据结构的动态演示,可以嵌入到MOOC的课程网站中,也可以独立在单一的网页中显示,从而搭建基于MOOC的数据结构课程动态演示系统。
\end{sectext}
%

% % !TEX root = ../o.tex
\chapter{进度安排}
\begin{chatext}
% !TEX root = ../o.tex
% c4t1.tex

\begin{center}
\setstretch{1.35}
\vspace{-20pt}
\vspace{-14pt}
\begin{longtable}{l r l l l l}
% \caption{算法可视化的效果}
\label{tab:3} \\
% \hline \multicolumn{1}{l}{积极方面} & \multicolumn{1}{l}{消极方面} \\ \hline
\endfirsthead
% \multicolumn{2}{r}%
% {接表 \ref{tab:3}} \\
% \hline \multicolumn{1}{l}{积极方面} & \multicolumn{1}{l}{消极方面} \\ \hline
\endhead
% \hline \multicolumn{2}{r}{待续}
\endfoot
% \hline
\endlastfoot

2014年 & 12月 & -- & 2015年 & 3月 & 查阅文献、研究课题、完成文献翻译和开题报告\\
2015年 & 3月 & -- & 2015年 & 4月 & 进行需求分析、设计系统\\
2015年 & 4月 & -- & 2015年 & 5月 & 实现系统、完成测试\\
2015年 & 5月 & -- & 2015年 & 6月 & 完成毕业论文、参加答辩\\
\end{longtable}
\end{center}

\end{chatext}
\vspace{-44pt}

\chapter{引言}
\section{背景}
\subsection{国内}
\subsection{国外}
% !TEX root = ../a.tex
\acknowledgment

\nocite{*}
\bibliography{a.bib}
\end{document}


% 毕业设计期间必须做出工程产品一部分或相对完整的工程原型系统。现代软件工程非常强调撰写文档,在开发期间,各阶段都要撰写文档。毕业论文属技术报告型,报告本课题所依据的原理、规范和实现模型所需的环境支持,总体实现方案,各(子)部分的设计与实现,与外部的接口等。但不等于文档集合,毕业论文应该带有论证性。如这个课题当今有几种实现方案、为什么要选择此种方案、本方案有何优缺点等。
% 论文要突出重点,对核心、独创部分尽可能地详细。要强调系统性,即从分析、设计、实现到测试,每个阶段的重点技术要清楚。特别是测试结果和使用情况是工程型课题最为增色之处,切不可略去。
% 对工程型论文,强调应用性。例如完成一个不太大的实际项目或在某一个较大的项目中设计并完成一个模块(如应用软件、工具软件或自行设计的板卡、接口等等),然后以工程项目总结或科研报告、或已发表的论文的综合扩展等形式完成论文。
% 论文应重点收集整理课题的立题背景,需求分析,平台选型,使用的开发工具,系统体系结构,程序模块调用关系,数据结构,算法,实验或测试等内容 。 论文结构一般安排如下:
% (1)引言或背景 (概述题目背景,实现情况,自己开发的内容或模块) 。
% (2)一般谈课题意义,综述已有成果,如“谁谁在文献某某中做了什么工作,谁谁在文献某某中有
% 什么突出贡献”,用“但是”一转,分析存在问题,引出自己工作必要性、意义和价值、创新点和主要思
% 想、方法和结果。然后用“本文组织如下:第一章....,第四章.....” 作为这段结束。
% (3)课题的分析设计(包括概要设计和详细设计,应强调系统的整体性,重点描述项目的整体框架,
% 功能,开发工具,突出自己工作在整体中的位置,对概要设计的细化等。这部分内容是全文的重点)。
% (4)主要实现及功能的描述(包括模块调用关系,数据结构,算法说明,依据内容多少此部分可安
% 排两到三节)
% 。
% (5)实验或测试(描述数据库设计结果,代码开发原理和过程,实现中遇到和解决的主要问题,系
% 统的测试,今后的维护和改进等)。
% (6)总结(全文总结及展望)。
% 这类工作应进行系统演示。
