% !TEX root = ../../t.tex
\chapter{引言}
\begin{sectext}
要教授《数据结构与算法》(此后用《算法》指代)这门计算机科学中具有代表性的课程,教授、讲师、导师或者老师(此后用老师指代)都会用实际例子演示某个算法如何执行。通常的教学方法有如下几种:
\begin{itemlist}
\item 把一些静态图片或图表的例子提前打印在材料、课本或者PowerPont(或者其他类似的演示软件)的幻灯片中。如果要效果稍微好一点,就使用准备好的幻灯片动画来说明算法步骤。这种方法的缺点在于老师控制动画的速度可能会太快。另外,也很难展示其他没有嵌入的例子或者是学生当场提出的例子;老师只能直接手工画在PowerPoint的幻灯片上(用鼠标或者铁笔)或者用手在黑板上画(方法2)。

\item 老师在黑板上画出例子。不同于PowerPoint,老师不会过快跳到下一张幻灯片,这样手绘的方法让学生们能够以正常人类(学生)的节奏消化算法步骤,但画出算法的每一步可能需要一些时间。而缺点就在于这样的人工手绘太消耗时间,特别是在算法执行或准备出错的时候。同时如果算法复杂或者老师太过着急都容易出现错误。而学生们理解算法的同时,为了课后复习也会记录下老师的例子。这样一来就有冲突,学生一边需要记忆算法,另一方面有手动记录例子,如果错过了课程可能会恶化学习效果。

\item 老师只是想学生们指出可以去某些现有的算法动画网站(通常是由独立开发者创建)。其中一些网站能够有效教授算法,这些内容会在第二节中展开。但是这种方法的缺点在于大部分这些网站只题够一些经典算法的可视化。因此,如果老师对于一个算法提供一个网站,另一个算法有提供其他的网站,学生们就会因为网站不同的操作和布局感到难以理解。这些网站中各种不同的用户界面加大了学生学习的难度。一些学生可能还存在惰性现象而没有主动访问这些网站。同时,如果在教授一个少见的非经典算法时,没有网站提供可视化,那么这种方法也不可取。
\end{itemlist}

综上,教授算法更好的方法在于既能够提供带有动画的典型而且提前准备好的例子(用来初次介绍概念),又能供结合二至三个学生自己输入的例子。要使这样的方法高效需要满足两个条件:即时的例子必须在动画过程中保持结果一致而且无差错;这样的方法不能占用太多时间。
我们决定通过浏览器搭建一个基于网络的平台来实现可视化,这样全世界就有更多的人能够接触到算法知识。几乎现在所有的设备——台式电脑、笔记本,甚至是智能手机,全都带有内置浏览器,例如火狐浏览器、谷歌浏览器、苹果浏览器等等,这些都原生支持HTML5。因此无需安装任何程序,也没有跨平台的问题。

交互式而非静态可视化是因为学生对于算法可以获得更加深刻的理解。我们相信如果学生能够自己输入一些数据,那么就更加有利于理解算法的执行。

随着越来越多的可携带设备支持网络连接:笔记本、平板电脑和智能手机,许多学生可以在授课或者听讲老师的例子时用自己的设备访问我们的在线可视化网站,在课后也可以按照自己的节奏巩固算法执行过程。

在教学方面,有了这样的可视化网站,测验中许多只测试学生算法如何运行的问题也会减少。学生会更期待了解算法如何运行,所以这样的问题也变得多余起来。同时,老师也能够集中精力,出一些算法高级用法的题目。

统一的界面有利与学生学习。毫无疑问,一个典型的计算机科学课程会涉及一个算法以上。有了统一的界面,学生在使用的时候不会感到陌生。同时在整个学期中只需要将一个网页保存为书签,相比保存一大堆不同的网站更加简便。学生也不需要为了找到老师提供的网站而查阅课程的幻灯片或者材料。所有多余的不便都会对学生产生惰性,很可能导致学生偷懒,不愿理解算法含义。

这篇论文的结构如下:在第二节中,我们具体分析当前的一些算法可视化网站,算法选自''高级编程2''(Halim和Halim,2011)。这本高级编程手册有本论文的第一位作者和最后一位作者一同编写。这本书讨论了大量算法,其中许多算法都还没有可视化。同时我们还讨论了现有可视化的缺点,评判的标准是基于我们所想要达到的目标,也就是:可视化非经典算法,统一界面,交互输入用户提供的数据,兼容各设备端的浏览器,特别是智能手机。在第三节中,我们讨论了可视化系统的细节。在第四节中则汇报了来自新加坡过来立大学的31名学生最初的反馈,这些学生在2011--2012学年的第二学期使用了可视化系统的第一个版本。在第五节,就可视化系统的效果列出一些初步得到的结论,最后通过计划近期不断加强可视化系统总结全文。
\end{sectext}
