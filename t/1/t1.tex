% !TEX root = ../../t.tex
\begin{center}
\setstretch{1.35}
\vspace{-20pt}
\begin{longtable}{p{4cm} p{8cm} l l l}
\caption{单一可视化(截至2012年4月所有URL地址均可以访问)}
\label{tab:1} \\

\hline \multicolumn{1}{l}{主题} & \multicolumn{1}{l}{算法可视化网站地址} & \multicolumn{1}{c}{N} & \multicolumn{1}{c}{H} & \multicolumn{1}{c}{O}\\ \hline
\endfirsthead

\multicolumn{5}{r}%
{接表 \ref{tab:1}} \\
\hline \multicolumn{1}{l}{主题} & \multicolumn{1}{l}{算法可视化网站地址} & \multicolumn{1}{c}{N} & \multicolumn{1}{c}{H} & \multicolumn{1}{c}{O}\\ \hline
\endhead

\hline \multicolumn{5}{r}{待续}
\endfoot

\hline
\endlastfoot

{各种线性数据结构} & \url{apbrwww5.apsu.edu/myersb2/linked_list_animation.htm} & & H & O \\
& \url{www.u-www.cosc.canterbury.ac.nz/mukundan/dsal/StackAppl.html} & & H & O \\
{各种排序算法} & \url{www.cs.auckland.ac.nz/∼jmor159/PLDS210/Java/q_sort/tqs_new.html} & & H & O \\
& \url{www.cse.iitk.ac.in/users/dsrkg/cs210/applets/sortingII/mergeSort/mergeSort.html} & & & O \\
& \url{www.cs.oswego.edu/∼mohammad/classes/csc241/samples/sort/Sort2-E.html} & & H & O \\
位掩码或位操作 & (见4.1) & N & & \\
{基本的二叉搜索树(表格)} & \url{aleph0.clarku.edu/~achou/cs102/examples/bst_animation/BST-Example.html} & & & O \\
& \url{www.cs.jhu.edu/~goodrich/dsa/trees/btree.html} & & & O \\
& \url{www.csc.liv.ac.uk/~ullrich/COMP102/applets/bstree/} & & & O \\
{平衡树(AVL)} & \url{webdiis.unizar.es/asignaturas/EDA/AVLTree/avltree.html} & & & O \\
& \url{www.site.uottawa.ca/~stan/csi2514/applets/avl/BT.html} & & & O \\
& \url{www.cs.jhu.edu/~goodrich/dsa/trees/avltree.html} & & & O \\
{堆(优先队列)} & \url{nova.umuc.edu/~jarc/idsv/lesson2.html} & & & O \\
& \url{www.cs.auckland.ac.nz/~jmor159/PLDS210/heaps.html} & & & O \\
{图数据结构(邻接矩阵或邻接表)} & (\url{nova.umuc.edu/~jarc/idsv/} 中有一部分相关主题) & N & & \\
& (见4.4) & & & \\
{并查集} & \url{research.cs.vt.edu/AVresearch/UF/} & & & O \\
& \url{www.cs.unm.edu/~rlpm/499/uf.html} & & & O \\
Segment tree & (我们的项目今后会增加这种算法) & N & & \\
树状数组 & (见4.3) & N & & \\
{N皇后问题} & \url{www.animatedrecursion.com/advanced/the_eight_queens_problem.html} & & H & O \\
& \url{yuval.bar-or.org/index.php?item=9} & & H & O \\
经典动态规划 & (今后会增加这种算法) & N & & \\
{图的遍历:BFS/DFS等} & \url{www.cs.sunysb.edu/~skiena/combinatorica/animations/search.html} & & H & \\
& \url{www.rci.rutgers.edu/~cfs/472_html/AI_SEARCH/SearchAnimations.html} & & H & \\
割点、桥、强连通分量 & (见4.4) & N & & \\
{MST (Prim算法)} & \url{www.unf.edu/~wkloster/foundations/PrimApplet/PrimApplet.htm} & & H & O \\
& \url{students.ceid.upatras.gr/~papagel/project/prim.htm} & & H & O \\
{MST (Kruskal算法)} & \url{www.unf.edu/~wkloster/foundations/KruskalApplet/KruskalApplet.htm} & & H & O \\
& \url{students.ceid.upatras.gr/~papagel/project/kruskal.htm} & & H & O \\
SSSP (Bellman Ford算法)& (见4.7) & N & & \\
{SSSP (Dijkstra算法)} & \url{www.unf.edu/~wkloster/foundations/DijkstraApplet/DijkstraApplet.htm} & & H & O \\
& \url{www.dgp.toronto.edu/~jstewart/270/9798s/Laffra/DijkstraApplet.html} & & & O \\
APSP (弗洛伊德算法) & \url{www.pms.ifi.lmu.de/lehre/compgeometry/Gosper/shortest_path/shortest_path.html} & & H & O \\
网络流(Edmonds Karp算法) & \url{felix-halim.net/research/maxflow/index.php} & & H & O \\
&\url{www.eecs.wsu.edu/∼cook/aa/lectures/applets/ek/MaxFlowHome.html} & & H & O \\
DAG算法 & (今后会增加这种算法,包括:拓扑排序、最短/长路径、路径数) & N & & \\
树上的算法 & (今后会增加这种算法,包括:最短/长路径、所有点对最短路径、最近公共祖先) & N & & \\
二分图算法 & \url{www-b2.is.tokushima-u.ac.jp/~ikeda/suuri/main/index.shtml} & & H & O \\
字符串匹配 & \url{cgjennings.ca/fjs/index.html} & & & O \\
{后缀树} & \url{illya-keeplearning.blogspot.com/2009/06/suffix-trees-java-applet.html} & & & O \\
后缀数组 & \url{felix-halim.net/pg/suffix-array} & & & \\
多边形算法 & (见4.9) & N & & \\
% \multirow{3}{*}{\begin{minipage}{4cm}二维凸包 (Graham扫描算法,Andrew单调链算法,Jarvis卷包裹算法)\end{minipage}} & \url{www.cs.princeton.edu/courses/archive/spr10/cos226/demo/ah/GrahamScan.html} & & H & O \\
% \multirow{3}{*}{\parbox{4cm}{二维凸包 (Graham扫描算法,Andrew单调链算法,Jarvis卷包裹算法)}} & \url{www.cs.princeton.edu/courses/archive/spr10/cos226/demo/ah/GrahamScan.html} & & H & O \\
凸包 (Graham扫描、Andrew单调链、Jarvis卷包裹算法) & \url{www.cs.princeton.edu/courses/archive/spr10/cos226/demo/ah/GrahamScan.html} & & H & O \\
&\url{nms.lcs.mit.edu/ ∼ aklmiu/6.838/convexhull/} & & & O \\
&\url{www.cs.unc.edu/∼snoeyink/demos/ch/Jarvis.html} & & & O \\
\end{longtable}
\end{center}

\vspace{-5pt}
