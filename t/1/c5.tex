% !TEX root = ../../t.tex
\chapter{初步结论、未来展望、致谢}
\begin{sectext}
在本论文中,我们展示了一共九个算法可视化,这些可视化都有一致的风格、界面,由HTML5技术和Javascript编写支撑,同时还支持用户交互。31名学生中一半以上表示有助于学习,而其余则表示效果一般。

这个可视化项目还是一个正在不断完善的项目。现在,可以访问 \url{www.comp.nus.edu.sg/~stevenha/visualization} 查看该项目。这个URL地址可能以后会有所变化,但下面的一些关键词有助于重新在网上找到该项目:``算法''、``可视化''、``高级编程''、``SoC''、``NUS''以及作者的名字。

之后,我们有以下的计划:
\begin{itemlist}

\item 强化现有可视化的功能:增加最小树形图(树状)的朱刘、Edmonds算法(朱永津和刘振宏,1965;Edmonds,1967)、各种网络流模型和变形(例如:二分匹配、最小费用流等)和一些多边形上的算法。

\item 增加更多经典算法,完善算法可视化库,现在已经实现了算法包括:DAG上的算法(例如:最短路、最长路、路的数量)、树(例如:最近公共祖先等)。

\item 增加更多网上没有的非经典算法的可视化,例如线段树、Edmonds的最大匹配算法(Edmonds,1965),Hopcroft Karp的二分匹配算法(Hopcroft Karp,1973),后缀数组以及后缀树表示(Manber和Myers,1991)等。

\item 优化用户界面,使可视化更加对用户更加友好(尤其是在可视化的同时添加源代码高亮,以及添加``后退''功能)。

\item 提供更多教学使用场景,例如弹出对话框,测试下一步应该是怎样执行。

\item 扩展可视化系统,让其他开发者使用我们的API
\end{itemlist}

我们希望该可视化项目能够在世界各地的大学、计算机院系中得到使用。

本论文所有的作者非常感谢新加坡国立大学发展教学与学习中心(CDTL,NUS)提供该可视化项目的启动资金。
\end{sectext}
