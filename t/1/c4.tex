% !TEX root = ../../t.tex
\chapter{学生反馈}
\begin{sectext}
新加坡国立大学第二学期中(2012年1--2月),在第三节中介绍的算法可视化首次用于两种模式:
\begin{itemlist}
\item CS3233——高级编程(24名学生),任课老师为第一位作者。使用到的可视化包括:位掩码、位表示、DFS、BFS、MST、SSSP、最大流和多边形凸包。

\item CS2020——数据结构和算法速成(42名学生),任课老师为第一位作者的同事,使用到的可视化包括:堆、位表示、DFS、BFS、MST、SSSP。
\end{itemlist}

我们对两个班的学生使用以下调研来测量这些可视化带来的主观效果。调研于2012年3月展开,两个班的学生都各自看过上述列举的可视化。其中第一位作者所带CS3233课程的班上所有24名(100\%)学生都表示广泛地使用过。而参加CS2020课程的学生中只有7名(16.7\%)学生表示没有经常使用。我们把这两组学生的结果分成下面的两列,可以看出整体上两组学生的反馈非常相近。
\end{sectext}
\section{多项选择题(MCQ)}
\begin{sectext}
% !TEX root = ../../t.tex
\begin{center}
\setstretch{1.35}
\vspace{-20pt}
\begin{longtable}{p{5cm} p{4cm} p{4cm}}
\caption{MCQ结果}
\label{tab:2} \\

\hline \multicolumn{1}{l}{多项选择题} & \multicolumn{1}{l}{CS3233(24名受访者)} & \multicolumn{1}{c}{CS2020(7名受访者)} \\ \hline
\endfirsthead

\multicolumn{3}{r}%
{接表 \ref{tab:2}} \\
\hline \multicolumn{1}{l}{多项选择题} & \multicolumn{1}{l}{CS3233(24名受访者)} & \multicolumn{1}{c}{CS2020(7名受访者)} \\ \hline
\endhead

\hline \multicolumn{3}{r}{待续}
\endfoot

\hline
\endlastfoot

1、算法可视化是否帮助你更好地理解算法? & 有 (13/54.1\%)\par 没有 (0/0.0\%)\par 一般 (11/45.8\%) & 有 (5/71.4\%)\par 没有 (0/0.0\%)\par 一般 (2/28.5\%) \\
2、你什么时候使用算法可视化?(多选) & 课上 (8/33.3\%)\par 课后 (4/16.6\%)\par 考试前 (14/58.3\%)\par 其他 (6/25.0\%) & 课上 (1/14.2\%)\par 课后 (0/0.0\%)\par 考试前 (5/71.4\%)\par 其他 (3/42.9\%) \\
3、你如何访问算法可视化的网站?(多选) & 台式电脑 (14/58.3\%)\par 笔记本 (14/58.3\%)\par 平板电脑 (3/12.5\%)\par 智能手机 (5/20.8\%) & 台式电脑 (4/57.1\%)\par 笔记本 (6/85.7\%)\par 平板电脑 (0/0.0\%)\par 智能手机 (1/14.2\%)\\

\end{longtable}
\end{center}
\vspace{-42pt}

下面是几个对于问题1回答理由的节选:

有:对于视觉来说很直观;有助于解释概念;可以自己输入不同的数据;有助于一开始了解算法的执行;以前一直是自己手画这些例子。

一般:已经通过授课、书本、源代码了解过算法;之前已经掌握;更希望通过分析证明而不是可视化的方式来展示算法的执行;可视化过程中(这是用于上课的版本)出现了错误,影响算法的理解。

值得一提的是在32一个受访者中没有一人选择``没有,算法可视化没有让我对算法理解地更深''。因此这些可视化可以看作是一种教学辅助工具,极大帮助了一部分学生,同时不会对其他学生产生不好的影响。

从第二个MCQ中我们可以看出学生在课堂上能够参与讨论。而最常见的使用方式则是在测试、测验前使用。这说明可视化主要用于自学和复习。

从第三个MCQ中我们可以看出一些学生使用智能手机或者平板电脑访问可视化网站。虽然使用人数上还是比较少,但是并不是每个人都有这些设备,我们还是相信这种情况在未来会有所改善。
\end{sectext}
\section{开放式问答题}
\begin{sectext}
我们也为同学们准备了一些开放式问答题。首先请学生列出这些算法可视化的好处和坏处。一些有代表性的反馈列在表\ref{tab:3}中。
% !TEX root = ../../t.tex
\begin{center}
\setstretch{1.35}
\vspace{-20pt}
\begin{longtable}{p{7cm} p{7cm}}
\caption{算法可视化的效果}
\label{tab:3} \\

\hline \multicolumn{1}{l}{积极方面} & \multicolumn{1}{l}{消极方面} \\ \hline
\endfirsthead

\multicolumn{2}{r}%
{接表 \ref{tab:3}} \\
\hline \multicolumn{1}{l}{积极方面} & \multicolumn{1}{l}{消极方面} \\ \hline
\endhead

\hline \multicolumn{2}{r}{待续}
\endfoot

\hline
\endlastfoot

算法可视化能直接给出正确答案(在修复程序错误后)。 & 算法可视化的程序错误在早期测试使用中会影响理解。\\
简介的用户界面,易于使用。 & 一些功能对用户还不够友好(还缺少合适的教程文档)。\\
用户能使用自己的例子:输入具体数据、绘制图和多边形。 & 在CS3233教学中需要的算法可视化不仅仅是上述的九种。\\
有助于更好地理解算法。 & 有了算法可视化可能导致学生不主动练习。\\
比徒手模拟更快。 & 可视化的每个步骤最好配上算法的(伪)代码。\\
\end{longtable}
\end{center}

\vspace{-42pt}


接着,我们还询问学生对于这样一个统一界面网站的看法,网站中不仅仅有上述列举的九种可视化,在将来还会有更多。一些节选回复如下:

好:不仅有利于这门课程(CS3233)也对NUS其他算法课程或其他大学有帮助。

好:统一的界面非常好。学生通常会像可视化那样在心里或者纸上模拟算法。

一般:这是个好主意。但是要是没有这样的网站问题也不大。

不好:我个人更喜欢自己动手模拟算法,这并不困难,而且也帮助我更好地记忆算法。但我相信肯定有其他人会喜欢这种工具。
\end{sectext}
