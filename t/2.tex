% !TEX root = ../t.tex
\mktitle{可视化在计算机科学教育中的作用}{Eric Fouh,Monika Akabar,Clifford A. Shaffer}
% !TEX root = ../../t.tex
\mkabstract{
计算机科学教学核心在于充分理解动态处理过程,例如实现算法的过程或者计算实体间的数据流动过程。这些动态处理过程不适合用文本、图像等静态媒介来解释,也难以在授课中传达。本文的作者基于现有记录,围绕推动教学的工具,研究了计算机科学教育中可视化的历史。其后还讨论了计算技术影响可视化工具发展的变化以及近期技术的转变推动在线超文本教科书的发展变化。
}{
算法可视化,数据结构可视化,程序可视化,电子教科书,超文本教科书
}

\begin{sectext}
计算机科学教学中,一方面计算机科学(CS)学生不断实践编程,在早期课程中就要学会编写中小型软件。这种根深蒂固的看法下,学生常常会误解早期非编程性的课程。另一方面,大部分其他专业的学生只需要一些实际的编程经验,而CS学生还必须学习整体知识,其中大部分的核心知识恰恰是抽象的。我们看不见也摸不着算法或数据结构,更不用说那些更高阶的概念,即包含算法和数据结构的计算机科学基础理论,其中许多内容涉及到动态处理的详细过程。特别是中期阶段(在第一年学习了算法的基本概念后开始学习CS中的特定研究方向)的学生,这些学生需要很长一段时间学习动态处理过程。排序算法、搜索算法、算法中的数据结构、算法分析技术都是计算机科学中级水平的家常便饭。即便是CS课程中的编程语言和计算机体系结构等相对具体的课程也涉及计算机中数据的动态流动。
文本、图像等静态媒介难以展示动态处理过程。在授课过程中,教师通常会一边讲解一边在展示板上画出变化过程,而大部分学生仍然难以理解这样详尽的解释。因此长时间以来,许多CS教师对动态化和可视化的概念解释饶有兴趣,自然而然希望通过触手可及的潜在教育基础设备计算机来传达本学科的核心知识。本文的作者讨论了CS教育中算法可视化(AV)和程序可视化(PV)的应用。
\end{sectext}
% !TEX root = ../../t.tex
\chapter{AV的早期应用}
\begin{sectext}
AV在CS教育中有一段长时间的历史,最早可以追溯到1981年的``排序总结''(Baecker,Sherman,1981)和BALSA系统(Brown和Sedgewick, 1985)。之后还出现了数百种AV应用供教育者开放使用,以及数百篇关于AV的论文(AlgoViz.org文献库,2011)。好的AV应用使算法形象化,用图形展示算法中不同的状态、状态过渡的动画,用自然抽象化的方式代替数据结构中的内存地址和函数调用。

BALSA(Brown,Sedgewick,1985)、Tango(Stasko,1990)、XTango(Stasko,1992),Samba(Stasko,1997)和Polka(Stasko,2001)等早期的系统由于实现技术难以在使用者中进行传播,这也是早期教育软件常有的问题。麻省理工学院的Athena项目(Champine,1991)不仅提供交互式课程软件,同时也通过X Windows桌面系统解决了传播问题。因此20世纪80年代末到90年代初的许多AV系统都建立在X Windows基础上。但是许多教育机构的实验室内由于工作站没有足够大的功率来运行X Windows,仍然难以使用这些课程软件。

Hundhausen、Douglas和Stasko首次对AV系统教学效果进行大规模的评估(2002),包括24个小规模的元分析实验,结果发现其中的11个实验中,使用AV技术的学生数据和使用另一种AV技术或完全不使用AV技术的学生数据有显著的差异(p. 265),另外10个实验则没有明显结果,还有两个有明显结果,但是使用AV技术的学生数据不准确,最后一个同样也有显著结果,但结果反而得到的是AV技术阻碍了学生学习。由于缺少统一的知识类型标准,实验难以衡量教学成果。一些关于AV技术的研究考察学生概念性或者陈述性知识(对算法概念的理解)以及过程性知识(对算法步骤和数据操作的记忆)。而在这些研究中,只有在考察过程性知识时能够得到最显著的结果。

Hundhausen等的论文中评论过AV系统(2002),这些系统根据其教学效果都存在不同的结果。XTango的实验中,仅仅使用教科书与同时使用教科书和可视化工具这两者之间的数据在考试结果上没有差异。另一个关于用XTango和Polka(Byrne,Catrambone,Stasko,1999)作为授课工具的实验中得到的结果是:相比仅仅使用教科书的学生,看过演示动画的学生能够更好地判断出算法的下一步。而在编程方面,使用BALSA II (brown,1988)实际操作并观看过算法演示动画的学生比没有的学生表现得更好。该实验的作者指出当前研究的AV系统都是Java版本前的变形,而且学生只能在计算机实验室中使用电脑。

也许Hundhausen等(2002)的元分析中最重要的成果在于(a)学生如何使用AV对教学效果有影响,而不是学生看到了什么样的AV;以及(b)AV技术能够吸引学生主动交互时最有效果。这种关键在于交互的观点很大程度上影响了AV之后的发展。
\end{sectext}

% !TEX root = ../../t.tex
\chapter{互联网时代下的AV}
\begin{sectext}
随着万维网和我们熟知的互联网的出现,20世纪90年代中期发生了许多改变。越来越多的教师和学生都在使用AV系统,大多是因为Java的广泛使用,还有一小部分是由于Javascript和Flash。也就大约在这个时期,越来越多的学生在校园里拥有自己的计算机。而从20世纪90年代末到21世纪初,一种新的基于JAVA的AV开发系统逐渐形成,而以前基于X Windows的AV系统在教育领域内几乎完全失去了影响力。

基于JAVA的AV系统包括JSamba(Stasko 1998)、JAWAA(Pierson,Rodger,1998)、JHAVE`(Naps,Eagan,Norton,2000)、ANIMAL(Rossling,Schuler,Freisleben,2000)和TRAKLA2(Korhonen等,2003)。这些系统的共同特点在于,作为开发AV的完整工具,能够为AV开发者提供一个编程框架。Java版本前的系统难以深入开发,虽说投入大量精力后还是能够进行开发的。相反,一些基于Java的开发系统(例如JAWAA)则注重可视化开发的便捷性,但不太重视交互性。

与Java广泛使用起到同样关键性的变化还在于出现了许多不依赖于开发系统的AV。这是因为Java不受制于操作系统或者其他类似的限制,同时互联网也利于发布软件。这就改变了开发者的自身角色。虽然JHAVE和ANIMAL按照传统开发系统,系统还是非常庞大。而其他许多项目则将AV打包为独立软件,不能用于其他AV开发中。大型非打包的系统包括数据结构浏览器(Data Structure Navigator,DSN)(Dittrich,an den Bercken,Schafer,Klein,2001),交互式数据结构可视化(Interactive Data Structure Visualization,IDSV)(Jarc,1999),算法动画(Stern,2001),可视化的数据结构(Galles,2006)和TRAKLA2(Korhonen等,2003)。

Java出现的另一个影响在于教师学生开始进行一次性的AV开发。编写AV对于学生而言是学好JAVA的一种途径,这在90年代末非常吸引人。遗憾的是大多数这些学生编写的AV难以发挥教学作用(Shaffer等,2010)。不过还是有一部分AV是通过几年持续努力而编写出的Java小应用程序,并且在某些课程上展示而受到广泛好评。例如二叉树之类(Binary Treesome)(Gustafson,Kjensli,Vold,2011),JFLAP(Rodger,2008),马里兰大学的空间索引示例(Spatial Index Demos)(Brabec,Samet,2003)和弗吉尼亚理工大学算法可视化研究小组(Virginia Tech Algorithm Visualization Research Group)(2011)。
\end{sectext}

% !TEX root = ../../t.tex
\chapter{交互带来的影响}
\begin{sectext}
从21世纪头10年的早期到中期出现了一系列影响深远的工作组报告,这些报告是例年ACM计算机科学教育大会创新技术的一部分(Naps等,2003b,Naps等,2003b,Rossling等,2006)。Naps等在报告中对Hundhausen等(2002)在AV中与学生积极交互的必要性方面加以补充,定义了一种AV与学生交互层级类型的分类,包括以下6种分类:

\begin{itemlist}
\item 无观看层级即不使用AV

\item 观看层级能显示AV并控制执行速度

\item 响应层级能回答与当前可视化内容有关的问题

\item 应变层级能执行输入的数据

\item 构建层级能可视化自己编写的算法

\item 演示层级能收集学生的反馈
\end{itemlist}

这种学生交互性分类的层次很接近Bloom层次(Bloom,Krathwohl,1956)。

Urquiza-Fuentes和Velazquez-Iturbide(2009)展开了一次新的元分析来评估AV系统在教学上的重要性,评估围绕已知的教育利益,以Naps等(2003b)定义的层级为衡量,发现与无观看层级相比观看层级并不能促进学习理解。这与之前Hundhausen等的结论(2002)是一致的。而且遗憾的是,Shaffer等(2010)在报告中提到大部分AV系统仍然处于观看层级。

Urquiza-Fuentes 和 Velazquez-Iturbide发现相比使用观看层级的AV系统,学生在响应层级上的交互能够促进知识学习。而应变层级比响应层级有更好的效果。构建层级能够编写AV,相比应变层级更能提高学习成果(虽然需要更多的时间和精力),但在知识学习上没有演示层级效果好。

Urquiza-Fuentes和Velazquez-Iturbide从成功的大型AV系统中引用了具体示例。JHAVE(Naps等,2000)AV系统由一个图形化界面、算法的文字信息和一些学习问题组成。JSamba(Stasko,1998)允许教师或学生编写脚本自定义数据结构。TRAKLA2(Korhonen等,2003)中包含了练习题,练习中学生通过自己执行单步执行演示熟悉算法,而系统也通过向学生提问算法的下一步来评估。Alice(Cooper,Dann,Pausch,2000)实现了3D虚拟空间中对象的可视化。jGRASP(Hendrix,Cross,Barowski,2004)允许教师开发具体的可视化以达到预期的目标(即对象的显示更贴近学生的喜好)

Urquiza-Fuentes和Velazquez-Iturbide还提到AV系统成功的共同要素,这些系统都有叙述内容和文本解释,可视化的步骤加以解释后可以在交互的观看层级上提高学习效果。而响应层级系统的特点在于获取学生行为的反馈,让学生在交互中表现出对于知识的熟练程度。另外要达到常时间使用系统,学生可以编写或自定义一个AV(构建层级的交互)。优秀的AV系统同样还支持一些高级特性,比如说用功能丰富的界面控制可视化。

Myller,Bednarick,Sutina和Ben-Ari(2009)研究了协作学习中交互的影响,对Naps等的交互分类进行补充,更清晰地捕捉学生的不同行为。研究补充了4种层级,其中包括介于观看层级和响应层级之间的对比层级和输入层级。这种扩展交互分类(EET)同样可以在使用可视化工具时指导学生互相合作,Myller等(2009)针对这种看法进行实验,在实验环节中让学生结对学习,并在不同交互层级上使用程序可视化工具BlueJ(Sanders,Heeler,Spradling,2001)和Jeliot(Levy,Ben-Ari,Uronen,2003),同时观察记录所有学生的交流,发现EET层级与学生交互量正相关。可见对于交互性是成功的人机合作与协作共进的重要因素之一。
\end{sectext}

% !TEX root = ../../t.tex
\chapter{一些可视化工具}
\begin{sectext}
在这一小节中,作者详细描述几个可视化工具或系统,这些工具和系统在推进教学上有所记录,而且由于大多数采用了观看层级以上的交互特性,在对照实验结果中显示了学生使用前后数据上的差异。
\end{sectext}
\section{TRAKLA2}
\begin{sectext}
TRAKLA2在所有芬兰大学的常规教学中得到应用,是最广泛使用的AV集合之一。TRAKLA2有许多响应层级交互式的优秀例子,可以控制算法中数据结构的表现形式,而学生可以通过拖放GUI元素``创建''数据结构。TRAKLA2的练习要求学生操作数据结构的状态从而得到某些结果,比如说要创建树形结构,学生可能不断地拖放新元素到树中的指定位置。或者学生可以通过观察单步执行(也称之为参考答案)来理解算法。为了提供交互式的反复练习,大部分练习都有文本教程和解释算法的伪代码。

Laakso等(2005)在芬兰2所大学说明了数据结构与算法课程中使用TRAKLA2练习的情况。TRAKLA2练习贯穿于课堂(封闭实验室)练习、补课以及上课环节中,同时也出现在期终测试(占20\%)和期中测试(其中TRAKLA2练习5题,占50\%)。课程中引入TRAKLA2的一年后学生的上课积极性全面提高,不再仅仅是课堂练习的部分。课堂练习的平均表现从54.5\%提高到60.3\%(完成的练习数量)。学生通过在线调查反馈对于TRAKLA2的看法,80\%认为很好。而在使用一年后,认为TRAKLA2非常适合教学的学生数量显著上升,94\%认为其中的练习加快了学习过程。另一方面,学生也不介意获取工具和练习算法的方式,但更倾向与一种混合(在线课堂与传统课堂)的体验。这一结果同Levy和Ben-Ari(2007)得出的AV系统应该与现有课堂结合起来相一致,之后会再讨论到这点。

在成功的AV中,易用性对学习目标起到关键作用。通常学生在使用AV系统时需要更多的时间。但这可能是因为由于系统缺乏易用性(Pane,Corbett,John,1996)学生需要一定时间习惯。而至于TRAKLA2,有研究通过一系列调查问卷,并在一个易用的实验室进行观察(Laakso等,2005)发现并没有严重的易用性问题。学生需要在15分钟内完成所有练习,其中80\%的时间用于解题,而14\%的时间用于熟悉系统界面。这项重要的发现表明通常习惯给定的AV需要短则几分钟长则一、两个小时的时间。

TRAKLA2也用于评估EET对于学生表现的影响。Lassko,Myller和Korhonen(2009)研究了结对学习时使用EET应变层级的学生是否比使用EET对比层级的学生表现更好。第一年学生分成两组,每组都使用相同的文本材料。控制组(EET应变层级)使用TRAKLA2练习,对比组(EET对比层级)只是观察有同样信息的AV。所有的学生都独立参加了试前测试并自由结对,之后用45分钟学习材料并结对(使用纸和铅笔)解题。同时所有的学生都需要参加试后测试,并且用视频记录其过程。从第一个对试前测试和试后测试的分析中不能得到两组之间数据的显著差异,但从第二个对视频的分析中可以得到:一些在控制组的学生并非按实验预期一样使用TRAKLA2,只是看着参考答案而没有选择去解决TRAKLA2中的练习,因此这也是属于EET对比层级。于是控制组内使用对比层级交互的同学重新作为第三组,再分析得分,最后得到在控制组中使用应变层级交互的学生比使用对比层级交互的学生表现更好。
\end{sectext}
\section{JHAVE}
\begin{sectext}
JHAVE(Naps等,2000)一种AV开发系统,旨在相对简化AV开发者制作带有内置评估功能的动态幻灯片,可以在演示过程中向用户弹出问题。JHAVE的界面包括可视化面板和伪代码面板,通常还有一段关于算法的简短文字教程。信息页面中可以导入图片,用于显示算法的流程图。JHAVE内有非常多的AV,已经得到广泛使用。

Lahtinen和Ahoniemi(2009)展开了``快速启动''实验,向没有编程经验的同学开设可视化的CS1课程,并基于以下三点:

\begin{itemlist}
\item 课程从一个真实的编程问题入手,在解决问题中解释数据结构、伪代码、流程图和编程思想。课堂中使用可视化工具作为教学材料。

\item 用伪代码和流程图表示问题的解决方案,伪代码和英语描述类似。

\item 学生与算法(测试、调试等)进行互动。
\end{itemlist}

Lahtinen和Ahoniemi(2009)认为大多说AV系统假定学生已经很熟悉系统使用的编程语言。而这个实验是针对几乎没有任何编程经验的学生,就要用到没有语法限制的AV系统,其灵活性可以显示系统中算法的流程图,所以选择了JHAVE作为可视化工具。快速启动实验要求学生在课程初设计一个算法,称为``不成熟的算法''。之后对可视化后的算法进行测试并修改,最后在JHAVE的可视化中达到``成熟的算法''。实验中,系统与学生的交互性覆盖了观察、响应、应变、构建和演示层级。对于快速启动/JHAVE的评估表明86\%的学生认为可视化有助于学习,相比有编程经验的学生,没有编程经验的学生认为更有利于学习。这次有趣的实验把AV整合到教学方法中,JHAVE贯穿学生的整个课程,由于系统的可伸缩特点,将接近英语的伪代码和图像构造的算法流程图打包整合到JAVA小应用程序中,这也提供了一种使用AV工具的新途径。

JHAVE-POP(Furcy,2007)是JHAVE的一个插件,用于练习基本指针和链表操作,可以根据用户输入的C++或JAVA的代码片段,在程序声明执行时一步一步可视化地生成内存中的内容。学生所反馈的调查问卷(Furcy,2009)显示``JHAVEE-POP具体化指针的形象,加深了学生的理解'',同时JHAVE-EPOP``能够帮助学生理解并调试程序。''在这一小节中,作者详细描述几个可视化工具或系统,这些工具和系统在推进教学上有所记录,而且由于大多数采用了观看层级以上的交互特性,在对照实验结果中显示了学生使用前后数据上的差异。
\end{sectext}
\section{ALVIS}
\begin{sectext}
ALVIS(Hundhausen,Brown,2005)是一个程序开发环境,使用SALASA类伪代码语言编写程序,支持分镜功能。Hundhausen和Brown(2008)为了让有过一学期编程经验的学生在交互过程中覆盖五个层级而开展了教学实验。在实验中,学生结对使用ALVIS开发课堂中学到的算法的可视化,与同学和老师讨论并展示结果。Hundhausen和Brown把学生分为两组来评估系统,其中一组使用ALVIS另一组使用文本工具(钢笔、纸头、胶带等)。所有学生在文本编辑器(文本工具小组)或者ALVIS上使用SALSA语言编写,由摄像头进行录像。在实验结束后,对其所用到的工具进行回收,同时所有学生都需要描述自己的感受,使用文本工具的学生还需要参加采访。两组学生都用大部分时间编码,相比使用ALVIS的学生,使用文本工具的学生与助教讨论的时间更多,但所编写的每个算法代码中的错误却多出将近一倍。Hundhausen和Brown认为无论用哪种工具学生都可以学到互相讨论和批判思考。整个实验发现ALVIS能够帮助学生快速编写出语义错误较少的代码,同时可以调动学生参与到课堂,积极讨论算法。
\end{sectext}
\section{弗吉尼亚理工大学哈希教程}
\begin{sectext}
弗吉尼亚理工大学哈希教程(弗吉尼亚理工大学算法可视化研究小组网站,2011)并不只是一个简单的AV,教程涉及CS中最重要的主题,哈希搜索的概念,内容多达一整本教科书,其中包含的AV(以Java小应用程序的形式)嵌入在HTML文本中进行显示。除了展示基本的哈希处理过程,学生可以用附加的小应用程序观察教程中不同算法相对的性能差异。

2008年和2009年在大二年级开设了不同的数据结构与算法课程,学生分别都学习了哈希算法。第一种课程中,使用一周的标准课本与授课,与前几周一样,而在另一种课程中用一周的时间用同样的课本学习带有AV的网上教程,课本在两个对比课程中是完全一样的,但网上教程在AV中补充了更多的文字。

在两次实验中,两门课程都在一周结束后进行测验,测验结果表明使用网上教程的课程相比标准授课数据上有明显的优势。也就是说网上教程不仅仅和授课一样有效(对于远程学习意义深远),而且在恰当的交互体验下,计算机教学比授课教学(被动形式)效果更好。但是研究还需要考虑在课堂听讲和在实验室学习教程这样的对比环境对结果有多大的影响。如果一个学生只是自己阅读材料,学习AV,由于不同的自律表现,不能保证有充足的时间和精力,这样试验结果可能就完全不一样了。同样在对比环境中,相比只是自己阅读课本,阅读课本后再听课也可能会产生不同的结果。
\end{sectext}
\section{AlViE}
\begin{sectext}
和JHAVE一样,AlViE(AlViE,2011)是一个测试后的算法可视化工具,可以在一边执行算法一边显示可视化内容。也就是说可视化由脚本驱动,脚本由不同的方式生成,包括工具程序执行的一个复杂算法或模拟。AlViE由Java编写,用XML描述执行算法中相关事件和数据结构。

Crescenzi和Nocentini(2007)在两年的数据结构和算法课程教学中反复使用AlViE,让学生与系统的交互层级达到最大。第一年的实验的交互覆盖了无观察、观察、构建和演示层级,课程系统地使用AlViE进行算法和数据结构的教学,回家作业要求在没有可视化工具下实现算法,期末项目要求使用AlViE编写特定算法的动画并想同学和老师展示。根据学生反馈,70\%认为期末测试中使用AlViE有用,30\%则认为非常有用。同时有90\%认为动画演示有助于理解算法流程,所有学生都觉得可视化具有教学价值。

第二年改进了之前的实验,基于实验出版了一本纸制教科书(Crescenzi、Gambosi,Grossi,2006),并加入了交互中的应变层级。这本书是AlViE的延伸,描述了算法和数据结构,其中的插图均来自AlViE的图形界面。阅读时可以看到所有算法的执行过程。现在意大利的几所大学已经采用了这本书和AlVie。
\end{sectext}
\section{Alice}
\begin{sectext}
Alice(Alice,2011)是一个3D交互式编程环境,通过简洁的图形化界面让初学者接触面向对象编程。用户可以创建虚拟世界并拖拽物体到主窗口中,也可以编写脚本控制物体。Alice操作简单,以讲故事的交互形式,赢得了许多CS学校的青睐,已经应用于不同层次的编程导论课程中,可面向CS专业、非CS专业、初中和高中学生。

Moskal、Luri和Cooper(2004)研究了使用Alice是否能够CS1的学生表现,具体的问题包括:Alice能否有效提高CS专业学生的升级率?Alice能否让学生相信自己的才能可以在CS领域取得成功?实验中所有学生都是CS专业,分为三组:(a)控制组,几乎没有编程经验并参加基于Alice课程的学生,同时;(b)对比组1,几乎没有编程经验但没有参加基于Alice课程的学生;(c)对比组2,有编程经验且没有参加基于Alice课程的学生。实验课程中,控制组的学生取得了3.0±0.8的GPA,对比组1的学生取得了1.9±1.3的GPA,控制组2取得了3.0±1.2的GPA。这样的结果有可能十分重要,因为过去相比有编程经验的新生,几乎没有编程经验的新生在CS课程中表现更差,升级率更低。过去两年的实验中,控制组的升级率达到88\%,控制组1达到47\%,控制组2达到75\%。对学生态度的评估显示没有编程经验也没有参加基于Alice课程的学生``在CS1课程后更加抵触计算机科学的创造性''(p. 78)。整个实验表明Alice能够有效提高学生表现和升级率,同时改善学生对CS的看法。
\end{sectext}
\section{Jeliot}
\begin{sectext}
Jeliot(Levy等,2003)用于高中Java编程的教学,是一种PV工具,可以直接观看抽象算法对应实际程序的操作而不是内嵌的展示图,还可以在界面中选择单步执行程序。学生使用时一边查阅程序源代码,一边观看自动生成的动画。PV系统用于生成可视化内容,展现实际程序的具体操作,而无需程序员写程序那样多的精力。但系统的缺点在于不能简单分离无关细节并聚焦算法的高阶部分或指定的关键部分。

Moreno、Sutinen、Bednarik和Myller(2007)用Jeliot中的可视化显示矛盾动画,认为这样可以暴露学生对编程概念的误解。矛盾动画扩展了交互方式,响应层级中要求学生指出当前动画中的错误,应变层级中要求学生纠正错误,构建层级和演示层级则要求学生构建一个矛盾动画并向同学展示。Moreno、Joy、Myller和Sutinen(2010)开发了Jeliot ConAn系统,系统基于Jeliot的AV系统,用于生成包含错误的动画。

为了强化学生的构思模型,Ma、Ferguson、Roper、Ross和Wood(2009)提出四阶段模型,用于检测并修正编程概念上错误的构思模型。

\begin{itemlist}
\item 准备工作:找到已有的错误构思模型

\item 感知错误:引发已有构思模型的冲突后修改模型

\item 构建模型:借助可视化构建正确的构思模型

\item 实际应用:(通常通过解决编程问题)测试新构建的构思模型
\end{itemlist}
为了测试构思模型,每个学生都登录系统,系统列出必须掌握的所有编程概念的大纲,每一条概念都有一系列习题,每一道习题都包含一个认知问题、有相关Jeliot系统中的可视化材料和最后一个测试概念理解的问题。如果学生在人质问题上回答错误,就需要运行附带的可视化动画,识别出自己的模型与程序执行之间的差异。在这一阶段,学生可以请教老师帮助深入理解算法,之后再运行相似的程序测试不同数据的同一个概念。实验从三个方面对模型进行评估:条件和循环、作用域和参数传递以及对象引用的赋值。对与所有44个学生中能够正确理解概念的人数,在条件和循环方面,实验前有23\%,实验后有73\%;在作用域和参数传递方面,实验前只有19\%,实验后则有85\%。Ma等(2009)发现四阶段模型在简单的概念上是正确的,但对于复杂概念难以得到准确结果(例如对象引用的赋值概念)。
\end{sectext}
\section{ViLLE}
\begin{sectext}
ViLLE(Rajala、Laasko、Kaila,Salakowski,2007)是芬兰的Turku大学开发的程序可视化工具,用于辅助教学或者可视化教师或学生创建的示例程序。系统支持多种多种编程语言(包括Java、C++和可扩展的伪代码),内置编辑器响应交互式的弹窗测试问答。学生可以跟踪程序状态和数据结构的变化。

为评估ViLLE的有效性,学生在第一年编程课程中使用系统(Rajala、Laasko、Kaila,Salakowski,2007)。实验将学生分为两组进行两小时的计算机上机。两组学生都参与了试前测试,回答程序输出、状态等问题后,一组学生在ViLLE下学习编程教程,而另一组则用教科书学习,最后同时参加了试后测试,题目同试前测试一样,额外还有两个关于使用给定语句完成程序的问题。实验在知识获取方面的结果并不明显,但是在使用ViLLE的小组中,之前没有编程经验的学生比有编程经验的学生获取的知识更多。虽然缩小了编程新手和老练程序员之间的差距,但最后测试总分上仍有明显的差异。另外在高中的ViLLE实验中(Kaila、Rajala、Laakso,Salakoski,2009),学生可以从Moddle课程管理系统获取所有的课堂材料。期中测试中,使用ViLLE的学生相比对比组表现更好(数据上很明显),认为ViLLE对于跟踪程序执行和学习编程技巧上非常有效。
\end{sectext}
\section{jGRASP}
\begin{sectext}
jGRASP(Hendrix等,2004)是一个完整的程序开发环境,拥有同步``对象浏览器''功能,将对象和数据结构状态可视化,现在已经应用于使用Java的CS2课程实验中。Jain、Cross、Hendrix和Barowski(2006)在Auburn大学进行一些列研究,分析了jGRASP的教学优势,即使用jGRASP的数据结构浏览器是否能让学生写出错误少的代码、准确检测并修改不是语法性质的错误。其中的两次实验在实验室内进行,分别把学生分为两组,一组只使用JGRASP调试器而另一组调试器和对象浏览器。实验要求学生实现单向链表的四个基本操作,结果显示在准确度上两组数据有明显差异,控制组的平均准确度为6.34,而对比组为4.48。研究认为在大部分(95\%)情况下,jGRASP能够提高准确度并减少编程实现数据结构的时间(pp. 35)。另外在第二次实验还要求学生在实现单向链表函数的同时,检测并修改Java程序中的错误(一共九个),控制组平均检测出的错误为6.8,成功改正的错误为5.6,新引入的错误仅为0.65,而对比组则为平均检测错出的误数为5.6,成功改正的错误为4.2,新引入的错误为1.3。但控制组平均使用时间更多一点,完成实验耗时88.23分钟而对比组为87.6分钟。
\end{sectext}
\section{JFLAP}
\begin{sectext}
JFLAP(Rodger,2008)可以辅助教学形式语言和自动机理论,尽管这些都是高年级课程,但系统已经用于新生第一学期的形式语言课程中。学生可以使用JFLAP创建并模拟可视化后的自动机,还可以分析语法中的字符串,将非确定有限自动机转换为确定性有限自动机。

在两年的实验中,第一年有12所大学,第二年有14所大学应用并研究JFLAP的效果(Rodger等,2009)。实验主要研究以下问题:JFLAP加快学习过程的效果有多大?JFLAP对于形式语言和自动机课程有哪些额外的作用?在第一年里,大部分教师将其用于回家作业和课堂展示;没有教师用于测验。结果显示学生中55\%在准备测验时使用系统,94\%有充足的时间学习如何使用系统,64\%表示系统有助于提高成绩,83\%表示系统相比书写更加方便,但50\%认为即便没有JSFLAP也能达到一样的效果。而在第二年里,试前测试和试后测试都有所减少,调查显示学生中,32\%在准备测验时使用系统的时间超过复习总时间的五分之一,29\%使用系统研究扩展问题,63\%表示课程更加有趣,72\%表示会主动参与课堂互动。
\end{sectext}

% !TEX root = ../../t.tex
\chapter{CS课程中AV的使用频率}
\begin{sectext}
CS教育会议和CS教育邮件在过去十多年里定期调查了教师对于AV的兴趣以及使用水平。Naps等(2003b)的报告中有三项2000年的研究,报告指出大部分(超过90\%)的观点支持使用AV,但那时大约仅有10\%表示已经在数据结构课程教学中使用AV,而37.5\%表示偶尔会在课堂中使用AV。

作者在SIGCSE'10(主要的CS教育会议,2010年3月在Milwaukee举办)会议中展开调查,研究教师对于AV及其课堂使用的态度。详细内容请参考Shaffer等(2011)。这次研究继承了Naps等(2003b)报告中2000年的研究以及与其研究相一致的结论,但没有调查AV的使用频率。结果显示43为受访者中有41人表示或强烈表示AV能帮助学习计算机科学,另外2人表示中立。但是仅有一半的人过去两年内在教学中使用过,而且回答``使用过''也不一定指的是使用所谓的AV,根据2000年的报告,可能是计算机中某种动态可视化软件。

调查的第三个问题涉及阻碍教学中使用AV的最大原因。与Naps等(2003a)、Naps等(2003b)以及Rosseling等(2006)的结论一致,大约一半的受访者表示找不到合适的AV,另一半则反映了结合使用AV的问题。总体上这些在教学技术领域有代表性的问题都难以解决(Hew,Brush,2007)。虽然给学生AV的补充材料不难,但是要把AV整合到授课、实验和作业中也不简单,一部分是因为教师没有时间改进课程,另一部分则因为AV和课本内容不一致。

课程中使用AV很大程度要看教师是否愿意整合到课本中。Levy和Ben-Ari(2007)研究了教师对于Jeliot的选择。实验中所有教师都很熟悉Jeliot但没有在教学中使用过AV,总体分为四组,要求其中一组在日常教学中使用一年,另外两组至少在习题中使用一次。实验结果将教师使用AV的感受分为四类:

两者和一:教师觉得对教学有用,频繁使用并整合到课程材料和活动中。

书本为主:教师觉得对教学有用,偶尔使用且不能很好地贴合课程,主要依赖课本

排斥:教师觉得对教学作用有限,不在课堂中使用

矛盾:教师支持系统,但除非必要否则不愿意在课堂中使用。

为了改善教师使用AV的体验,Levi和Ben-Ari认为系统设计者应该考虑把AV整合到教材等现有课程的材料,Naps等(2003b)还建议AV设计者创建对应的AV维护网站方便整合过程。

AlgoViz平台是教师获取AV的重要途径,为用户和开发者提供了AV相关的服务和资源,包括500种AV和大型的相关文献库。Algoviz试图处理上述调查结果中缺乏AV可用性的问题,允许向社区贡献自己的一份力。这份力可以是一个简单AV中所包含的信息,也可以是社区用户之间的直接交流。除了像论坛一样进行交流,还可以在社区中进行评级、推荐分享使用经验和方法。在之前的调查中教师关心的是这些教学资源质量如何、怎样使用,因此资源反馈信息与资源数量一样重要,但这些反馈通常不是来自真实课堂,所以业界报告也可以作为补充,提供一些实际数据支撑。

过去几年内,一些重要的技术性因素促使AV更易于在课堂中应用。相比以前,研究水平进步了,获取媒介也增加了互联网搜索和AlgoViz平台,从而聚集越来越多的AV(Shaffer等,2010)。现在课堂内外教师和学生有更多机会接触互联网,于是使用AV也就更加切实可行。例如,在弗吉尼亚理工大学,过去十几年要将互联网材料用教室里的计算机投影到屏幕上都十分困难,而在过去几年里这种情况已经普遍实现。相比过去,这让主流教师使用这样的技术更加有信心(Hew,Brush,2007)。在与教师讨论后,通过调查中的评论我们知道虽然互联网技术不断进步,但并非所有学校都拥有支持互联网展示的教室。

另一个推动AV使用的潜在原因在于笔记本电脑和移动设备的普及。在弗吉尼亚理工大学,所有工科专业都必须拥有一台平板电脑,而且大部分都有移动设备,例如智能手机、电子书阅读设备或iPad。其中非PC设备的缺点在于对许多课程软件涉及到的技术有所限制。所以虽说互联网已经在学生群体中普及,但还是存在限制,例如Java小应用程序无法在大部分非PC设备上运行。然而最近随着HTML5称为标准,交互式内容相对以前更容易开发,而且所有主流浏览器不需要任何插件也能支持新标准下开发的程序,并且适应不同的移动设备。AV开发者能够接触到更能多的用户,同时开发的课程软件也更便于获取。
\end{sectext}

% !TEX root = ../../t.tex
\chapter{未来的电子教科书}
\begin{sectext}
许多CS社区都想创造超文本教科书,也就是在线课本,其中整合了AV、练习题以及传统的文字图片。过去二十年里这项工作都有所进展,一边用更先进的技术推广AV,一边提高Bloom分类中学生与AV的交互水平。具体可以阅读Rosseling等(2006)、Ross和Grinder(2002)以及Shaffer、Akbar、Alon、Stewart和Edwards(2011)中关于定义和实现超文本课本的工作背景。

值得期待的是,终于可以在不久的未来凭借现在的技术完成这项工作。众所周知,HTML5开启了一个前所未有的新时代,计算机划时代的新功能在教学上可以主动的和学生交互并及时获得反馈。另外CS社区非常支持通用制作,这项工作将不仅仅是一本交互式电子课本,而且还能够制作课本,也就是说这项工作制定一种许可标准,并支持社区开发,最终允许教师修改电子书中的内容,选择主要部分、修改章节或者从其他书本中节选后合并成新的电子书。这种电子教科书的通用制作已经用于Connexions项目(2011)中,但现在Connexions提供的电子教科书并不能达到期望的AV交互水平。

根据前文提到的一些结果,相比新的教学内容或者AV整合到已有的章节中,直接使用完整的教学内容(完整的章节或者是一学期的课程)更加简单。对于一门新的课程,教师通常会选择一本新课本,并且准备一些上课笔记和辅助材料;对于之前教过的课程,教师还是会重新选择新课本,并且为不同的章节准备演示文稿。其中关键就在于选择一整块内容相当于完全替换一整块教学时间,这与在现有材料上压榨新内容或者讲述新技术不同。过去,大部分AV开发者都默认自己的AV将会整合到现有的某节课中,例如排序算法的AV。由于不按课本解释算法,这种AV最终只能整合到某节课,或作为课本的补充由学生自学,这样又回到刚才提到的问题,即AV与其他章节的材料不一致。一个完整的教学内容(包括AV)更易于替代现有教学章节。

这项工作最后还希望学生也加入电子教科书的共同制作理念。学生可以对特定章节、子章节或练习进行评论、贴标签、评估内容。教师在学生与系统交互过程中与学生互动,一起改进课程,关注对教学有效果的具体内容,一方面可以增加与学生的互动,另一方面可以促进完善教学。
\end{sectext}


\bibliography{}
\begin{chatext}
{
\leftskip 2em
\parindent -2em

AlgoViz.org Bibliography. (2011). Annotated bibliography [of AV literature]. Retrieved from \url{http://algo-viz.org/biblio}

Alice Web Site. (2011) Retrieved from \url{http://www.alice.org}

AlViE Web Site. (2011). AlViE 3.0 [software]. Retrieved from \url{http://alvie.algoritmica.org/}

Baecker, R., \& Sherman, D. (1981). \textit{Sorting out sorting}. Retrieved from \url{http://video.google.com /videoplay?docid=3970523862559774879}

Bloom, B. S., \& Krathwohl, D. R. (1956). \textit{Taxonomy of educational objectives: The classification of educational goals. Handbook I: Cognitive domain}. Harlow, England: Longmans.

Brabec, F., \& Samet, H. (2003). Maryland spatial index demos Web site. Retrieved from \url{http://donar.umiacs.umd.edu/quadtree/}

Brown, M. H. (1988). Exploring algorithms using Balsa-II. \textit{Computer}, 21(5), 14–36.

Brown, M. H., \& Sedgewick, R. (1985). Techniques for algorithm animation. \textit{IEEE Software}, 2(1), 28–39.

Byrne, M. D., Catrambone, R., \& Stasko, J. T. (1999). Evaluating animations as student aids in learning computer algorithms. \textit{Computers \& Education}, 33(4), 253–278.

Champine, G. A. (1991). \textit{MIT project Athena: A model for distributed campus computing}. Newton, MA: Digital Press.

Connexions Web site. (2011). Retrieved from \url{http://cnx.org/}

Cooper, S., Dann, W., \& Pausch, R. (2000). Alice: A 3-D tool for introductory programming concepts. \textit{Journal of Computing Sciences in Colleges}, 15(5), 107–115.

Crescenzi, P., Gambosi, G., \& Grossi, R. (2006). \textit{Strutture di dati e algoritmi}. Upper Saddle River, NJ: Pearson Education, Addison–Wesley.

Crescenzi, P., \& Nocentini, C. (2007). Fully integrating algorithm visualization into a CS2 course: A two-year experience. In J. Hughes, D. R. Peiris, \& P. T. Tymann (Eds.), \textit{Proceedings of the 12th Annual SIGCSE Conference on Innovation and Technology in Computer Science Education}, ITiCSE’07 (pp. 296–300), Dundee, Scotland.

Dittrich, J.-P., van den Bercken, J., Schafer, T., \& Klein, M. (2001). \textit{DSN: Data structure navigator}. Retrieved from \url{http://dbs.mathematik.uni-marburg.de/research/projec} \url{ts/dsn/}

Furcy, D. (2007). \textit{JHAVEPOP}. Retrieved from \url{http://jhave.org/jhavepop/}

Furcy, D. (2009). JHAVEPOP: Visualizing linked-list operations in C++ and Java. \textit{Journal of Computing Sciences in Colleges}, 25(1), 32–41.

Galles, D. (2006). Data structure visualization. Retrieved from \url{http://www.cs.usfca.edu/galles/visualization}

Gettys, J., Newman, R., \& Scheifler, R. (1989). \textit{The definitive guides to the X Window system} (Vol. 2) (Xlib Reference Manual for Version 11). Sebastopol, CA: O’Reilly.

Gustafson, B. E., Kjensli, J., \& Vold, J. M. (2011). \textit{Binary treesome Web site}. Retrieved from \url{http://www.iu.hio.no/∼ulfu/AlgDat/applet/binarytreesome}

Hendrix, T. D., Cross, J. H., II, \& Barowski, L. A. (2004, March 3–7). An extensible framework for providing dynamic data structure visualizations in a lightweight IDE. In D. T. Joyce, D. Knox, W. Dann, \& T. L. Naps (Eds.), \textit{Proceedings of the 35th SIGCSE Technical Symposium on Computer Science Education} (pp. 387–391), Norfolk, VA.

Hew, K., \& Brush, T. (2007). Integrating technology into K12 teaching and learning: Current knowledge gaps and recommendations for future research. \textit{Educational Technology Research and Development}, 55, 223–252.

Hundhausen, C. D., \& Brown, J. L. (2005). What you see is what you code: A radically dynamic algorithm visualization development model for novice learners. \textit{Proceedings of the 2005 IEEE Symposium on Visual Languages/Human-Centric Computing} (pp. 163–170), Dallas, TX.

Hundhausen, C. D., \& Brown, J. L. (2008). Designing, visualizing, and discussing algorithms within a CS1 studio experience: An empirical study. \textit{Computers \& Education}, 50, 301–326.

Hundhausen, C. D., Douglas, S. A., \& Stasko, J. T. (2002). A meta-study of algorithm visualization effectiveness. \textit{Journal of Visual Languages and Computing}, 13(3), 259–290.

Jain, J., Cross, J. H., Hendrix, T. D., \& Barowski, L. A. (2006). Experimental evaluation of animated-verifying object viewers for Java. In M. Burnett \& S. Diehl (Eds.), \textit{Proceedings of the 2006 ACM Symposium on Software Visualization} (pp. 27–36),
Brighton, UK.

Jarc, D. J. (1999). \textit{Interactive data structure visualization}. Retrieved from \url{http://nova.umuc.edu/jarc/idsv.}

Kaila, E., Rajala, T., Laakso, M.-J., \& Salakoski, T. (2009). Effects, experiences and feedback from studies of a program of course-long use of a program visualization tool. \textit{Informatics in Education}, 8(1), 17–34.

Kann, C., Lindeman, R. W., \& Heller, R. (1997). Integrating algorithm animation into a learning environment. \textit{Computers \& Education}, 28(4), 223–228.

Korhonen, A., Malmi, L., Silvasti, P., Nikander, J., Tenhunen, P., M ard, P., ... Karavirta, V. (2003). \textit{TRAKLA2}. Retrieved from \url{http://www.cs.hut.fi/Research/TRAKLA2/}

Laakso, M.-J., Myller, N., \& Korhonen, A. (2009). Comparing learning performance of students using algorithm visualizations collaboratively on different engagement levels. \textit{Educational Technology \& Society}, 12(2), 267–282.

Laakso, M.-J., Salakoski, T., Grandell, L., Qiu, X., Korhonen, A., \& Malmi, L. (2005). Multi-perspective study of novice learners adopting the visual algorithm simulation exercise system TRAKLA2. \textit{Informatics in Education}, 4(4), 49–68.

Lahtinen, E., \& Ahoniemi, T. (2009). Kick-start activation to novice programming—A visua-lization-based approach. \textit{Electronic Notes on Theoretical Computer Science}, 224, 125–132.

Levy, R. B.-B., \& Ben–Ari, M. (2007). We work so hard and they don’t use it: Acceptance of software tools by teachers. \textit{Proceedings of 12th Annual SIGCSE Conference on Innovation and Technology in Computer Science Education} (pp. 246–250), Dundee, Scotland.

Levy, R. B.-B., Ben–Ari, M., \& Uronen, P. A. (2003). The Jeliot 2000 program animation system. \textit{Computers \& Education}, 40, 1–15.

Ma, L., Ferguson, J., Roper, M., Ross, I., \& Wood, M. (2009). Improving the mental models held by novice programmers using cognitive conflict and Jeliot visualisations. \textit{SIGCSE Bulletin}, 41, 166–170.

Meier, A., Spada, H., \& Rummel, N. (2007). A rating scheme for assessing the quality of computer-supported collaboration processes. International \textit{Journal of Computer Supported Collaborative Learning}, 2(1), 63–86.

Moreno, A., Joy, M., Myller, N., \& Sutinen, E. (2010). Layered architecture for automatic generation of conflictive animations in programming education. \textit{IEEE Transactions on Learning Technologies}, 3, 139–151.

Moreno, A., Sutinen, E., Bednarik, R., \& Myller, N. (2007). Conflictive animations as engaging learning tools. In R. Lister \& Simon (Eds.), \textit{Seventh Baltic Sea Conference on Computing Education Research} (Koli Calling 2007) (pp. 203–206), Koli National Park, Finland.

Moskal, B., Lurie, D., \& Cooper, S. (2004). Evaluating the effectiveness of a new instructional approach. \textit{SIGCSE Bulletin}, 36, 75–79.

Myller, N., Bednarik, R., Sutinen, E., \& Ben–Ari, M. (2009). Extending the engagement taxonomy: Software visualization and collaborative learning. \textit{Transactions on Computing Education}, 9(1), 1–27.

Naps, T. L., Cooper, S., Koldehofe, B., Leska, C., Rossling, G., Dann, W., ... McNally, M. F. (2003a). Evaluating the educational impact of visualization. \textit{SIGSE Bulletin}, 35(4), 124–136.

Naps, T. L., Eagan, J. R., \& Norton, L. L. (2000). JHAV E—An environment to actively engage students in Web-based algorithm visualizations. \textit{SIGCSE Bulletin}, 32, 109–113.

Naps, T. L., Rossling, G., Almstrum, V., Dann, W., Fleischer, R., Hundhausen, C. D., ... Velazquez–Iturbide, J. (2003b). Exploring the role of visualization and engagement in computer science education. \textit{SIGCSE Bulletin}, 35(2), 131–152.

Pane, J. F., Corbett, A. T., \& John, B. E. (1996). Assessing dynamics in computer-based instruction. In R. Bigler, S. Guest, \& M. J. Tauber (Eds.), \textit{Proceedings of the ACM Conference on Human Factors in Computing Systems} (pp. 197–204), Vancouver, BC, Canada.

Pierson, W. C., \& Rodger, S. H. (1998). Web-based animation of data structures using JAWAA. \textit{SIGCSE Bulletin}, 30, 267–271.

Rajala, T., Laakso, M.-J., Kaila, E., \& Salakoski, T. (2007). ViLLe–A language-independent program visualization tool. In L. Raymond \& Simon (Eds.), \textit{Seventh Baltic Sea Conference on Computing Education Research (Koli Calling 2007)}, Koli National Park, Finland.

Rajala, T., Laakso, M.-J., Kaila, E., \& Salakoski, T. (2008). Effectiveness of program visualization: A case study with the ViLLE tool. \textit{Journal of Information Technology Education}, 7, 15–32.

Rodger, S.H. (2008). \textit{JFLAP Web site}. Retrieved from \url{http://www.jflap.org.}

Rodger, S. H., Wiebe, E., Lee, K. M., Morgan, C., Omar, K., \& Su, J. (2009). Increasing engagement in automata theory with JFLAP. In S. Fitzgerald, M. Guzdial, G. Lewandowski, \& S. A. Wolfman (Eds.), \textit{Proceedings of the 40th ACM Technical Symposium on Computer Science Education} (pp. 403–407), Chattanooga, TN.

Ross, R., \& Grinder, M. (2002). Hypertextbooks: Animated, active learning, comprehensive teaching and learning resources for the Web. In S. Diehl (Ed.), \textit{Software visualization} (Lecture notes in Computer Science 2269) (pp. 269–284). Berlin, Germany: Springer.

Rossling, G., Naps, T. L., Hall, M. S., Karavirta, V., Kerren, A., Leska, C., ... Velazquz–Iturbide, J. (2006). Merging interactive visualizations with hypertext-books and course management. \textit{SIGCSE BULLETIN}, 38(4), 166–181.

Rossling, G., Schuler, M., \& Freisleben, B. (2000). The ANIMAL algorithm animation tool. In J. Tarhio, S. Fincher, \& D. Joyce (Eds.), \textit{Proceedings of the 5th Annual SIGCSE/SIGCUE ITiCSE Conference on Innovation and Technology in Computer Science Education} (pp. 37–40), Helsinki, Finland.

Sanders, D., Heeler, P., \& Spradling, C. (2001). Introduction to BlueJ: A Java development environment. \textit{Journal of Computing Sciences in Colleges}, 16(3), 257–258.

Shaffer, C. A., Akbar, M., Alon, A., Stewart, M., \& Edwards, S. H. (2011). Getting algorithm visualizations into the classroom. \textit{SIGCSE Bulletin}, 11, 129–134.

Shaffer, C. A., Cooper, M., Alon, A., Akbar, M., Stewart, M., Ponce, S., \& Edwards, S. H. (2010). Algorithm visualization: The state of the field. \textit{ACM Transactions on Computing Education}, 10, 1–22.

Shaffer, C. A., Naps, T. L., \& Fouh, E. (2011). Truly interactive textbooks for computer science education. In G. Rossling (Ed.), \textit{Proceedings of the 6th Program Visualization Workshop} (pp. 97–103), Darmstadt, Germany.

Stasko, J. T. (1990). Tango: A framework and system for algorithm animation. Computer, 23(9), 27–39.

Stasko, J. T. (1992). Animating algorithms with Xtango. \textit{SIGACT News}, 23, 67–71.

Stasko, J. T., Badre, A., \& Lewis, C. (1993). Do algorithm animations assist learning? An empirical study and analysis. In S. Ashlund, K. Mullet, A. Henderson, E. Hollnagel, \& T. N. White (Eds.), \textit{Proceedings of the INTERACT ‘93 and CHI ‘93 Conference on Human Factors in Computing Systems} (pp. 61–66), Amsterdam, The Netherlands.

Stasko, J. T. (1997). Using student-built algorithm animations as learning aids. In C. M. White, C. Erickson, B. J. Klein, \& J. E. Miller (Eds.), \textit{Proceedings of the 28th SIGCSE Technical Symposium on Computer Science Education} (pp. 25–29), San Jose, CA.

Stasko, J. T. (1998). \textit{JSamba}. Retrieved from \url{http://www.cc.gatech.edu/gvu/softviz/algoanim/jsamba/}

Stasko, J. T. (2001). \textit{Polka animation system}. Retrieved from \url{http://www.cc.gatech.edu/gvu/softviz/par-viz/polka.html}

Stern, L. (2001). \textit{Algorithms in action}. Retrieved from \url{http://www.cs.mu.oz.au/aia/}

Urquizza–Fuentes, J., \& Velazquez–Iturbide, J. (2009). A survey of successful evaluations of program visualization and algorithm animation system. \textit{ACM Transactions on Computing Education}, 9(2), 1–21.

Virginia Tech Algorithm Visualization Research Group Web site. (2011). Retrieved from \url{http://research.cs.vt.edu/AVresearch/}

}

\end{chatext}
