% !TEX root = ../../t.tex
\chapter{未来的电子教科书}
\begin{sectext}
许多CS社区都想创造超文本教科书,也就是在线课本,其中整合了AV、练习题以及传统的文字图片。过去二十年里这项工作都有所进展,一边用更先进的技术推广AV,一边提高Bloom分类中学生与AV的交互水平。具体可以阅读Rosseling等(2006)、Ross和Grinder(2002)以及Shaffer、Akbar、Alon、Stewart和Edwards(2011)中关于定义和实现超文本课本的工作背景。

值得期待的是,终于可以在不久的未来凭借现在的技术完成这项工作。众所周知,HTML5开启了一个前所未有的新时代,计算机划时代的新功能在教学上可以主动的和学生交互并及时获得反馈。另外CS社区非常支持通用制作,这项工作将不仅仅是一本交互式电子课本,而且还能够制作课本,也就是说这项工作制定一种许可标准,并支持社区开发,最终允许教师修改电子书中的内容,选择主要部分、修改章节或者从其他书本中节选后合并成新的电子书。这种电子教科书的通用制作已经用于Connexions项目(2011)中,但现在Connexions提供的电子教科书并不能达到期望的AV交互水平。

根据前文提到的一些结果,相比新的教学内容或者AV整合到已有的章节中,直接使用完整的教学内容(完整的章节或者是一学期的课程)更加简单。对于一门新的课程,教师通常会选择一本新课本,并且准备一些上课笔记和辅助材料;对于之前教过的课程,教师还是会重新选择新课本,并且为不同的章节准备演示文稿。其中关键就在于选择一整块内容相当于完全替换一整块教学时间,这与在现有材料上压榨新内容或者讲述新技术不同。过去,大部分AV开发者都默认自己的AV将会整合到现有的某节课中,例如排序算法的AV。由于不按课本解释算法,这种AV最终只能整合到某节课,或作为课本的补充由学生自学,这样又回到刚才提到的问题,即AV与其他章节的材料不一致。一个完整的教学内容(包括AV)更易于替代现有教学章节。

这项工作最后还希望学生也加入电子教科书的共同制作理念。学生可以对特定章节、子章节或练习进行评论、贴标签、评估内容。教师在学生与系统交互过程中与学生互动,一起改进课程,关注对教学有效果的具体内容,一方面可以增加与学生的互动,另一方面可以促进完善教学。
\end{sectext}
