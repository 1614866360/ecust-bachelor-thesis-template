% !TEX root = ../../t.tex
\chapter{CS课程中AV的使用频率}
\begin{sectext}
CS教育会议和CS教育邮件在过去十多年里定期调查了教师对于AV的兴趣以及使用水平。Naps等(2003b)的报告中有三项2000年的研究,报告指出大部分(超过90\%)的观点支持使用AV,但那时大约仅有10\%表示已经在数据结构课程教学中使用AV,而37.5\%表示偶尔会在课堂中使用AV。

作者在SIGCSE'10(主要的CS教育会议,2010年3月在Milwaukee举办)会议中展开调查,研究教师对于AV及其课堂使用的态度。详细内容请参考Shaffer等(2011)。这次研究继承了Naps等(2003b)报告中2000年的研究以及与其研究相一致的结论,但没有调查AV的使用频率。结果显示43为受访者中有41人表示或强烈表示AV能帮助学习计算机科学,另外2人表示中立。但是仅有一半的人过去两年内在教学中使用过,而且回答``使用过''也不一定指的是使用所谓的AV,根据2000年的报告,可能是计算机中某种动态可视化软件。

调查的第三个问题涉及阻碍教学中使用AV的最大原因。与Naps等(2003a)、Naps等(2003b)以及Rosseling等(2006)的结论一致,大约一半的受访者表示找不到合适的AV,另一半则反映了结合使用AV的问题。总体上这些在教学技术领域有代表性的问题都难以解决(Hew,Brush,2007)。虽然给学生AV的补充材料不难,但是要把AV整合到授课、实验和作业中也不简单,一部分是因为教师没有时间改进课程,另一部分则因为AV和课本内容不一致。

课程中使用AV很大程度要看教师是否愿意整合到课本中。Levy和Ben-Ari(2007)研究了教师对于Jeliot的选择。实验中所有教师都很熟悉Jeliot但没有在教学中使用过AV,总体分为四组,要求其中一组在日常教学中使用一年,另外两组至少在习题中使用一次。实验结果将教师使用AV的感受分为四类:

两者和一:教师觉得对教学有用,频繁使用并整合到课程材料和活动中。

书本为主:教师觉得对教学有用,偶尔使用且不能很好地贴合课程,主要依赖课本

排斥:教师觉得对教学作用有限,不在课堂中使用

矛盾:教师支持系统,但除非必要否则不愿意在课堂中使用。

为了改善教师使用AV的体验,Levi和Ben-Ari认为系统设计者应该考虑把AV整合到教材等现有课程的材料,Naps等(2003b)还建议AV设计者创建对应的AV维护网站方便整合过程。

AlgoViz平台是教师获取AV的重要途径,为用户和开发者提供了AV相关的服务和资源,包括500种AV和大型的相关文献库。Algoviz试图处理上述调查结果中缺乏AV可用性的问题,允许向社区贡献自己的一份力。这份力可以是一个简单AV中所包含的信息,也可以是社区用户之间的直接交流。除了像论坛一样进行交流,还可以在社区中进行评级、推荐分享使用经验和方法。在之前的调查中教师关心的是这些教学资源质量如何、怎样使用,因此资源反馈信息与资源数量一样重要,但这些反馈通常不是来自真实课堂,所以业界报告也可以作为补充,提供一些实际数据支撑。

过去几年内,一些重要的技术性因素促使AV更易于在课堂中应用。相比以前,研究水平进步了,获取媒介也增加了互联网搜索和AlgoViz平台,从而聚集越来越多的AV(Shaffer等,2010)。现在课堂内外教师和学生有更多机会接触互联网,于是使用AV也就更加切实可行。例如,在弗吉尼亚理工大学,过去十几年要将互联网材料用教室里的计算机投影到屏幕上都十分困难,而在过去几年里这种情况已经普遍实现。相比过去,这让主流教师使用这样的技术更加有信心(Hew,Brush,2007)。在与教师讨论后,通过调查中的评论我们知道虽然互联网技术不断进步,但并非所有学校都拥有支持互联网展示的教室。

另一个推动AV使用的潜在原因在于笔记本电脑和移动设备的普及。在弗吉尼亚理工大学,所有工科专业都必须拥有一台平板电脑,而且大部分都有移动设备,例如智能手机、电子书阅读设备或iPad。其中非PC设备的缺点在于对许多课程软件涉及到的技术有所限制。所以虽说互联网已经在学生群体中普及,但还是存在限制,例如Java小应用程序无法在大部分非PC设备上运行。然而最近随着HTML5称为标准,交互式内容相对以前更容易开发,而且所有主流浏览器不需要任何插件也能支持新标准下开发的程序,并且适应不同的移动设备。AV开发者能够接触到更能多的用户,同时开发的课程软件也更便于获取。
\end{sectext}
