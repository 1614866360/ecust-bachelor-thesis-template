% !TEX root = ../o.tex
% \vspace*{-13pt}
\chapter{技术路线}
\begin{chatext}
研究将设计并实现一个基于MOOC的数据结构动态演示系统,采用web技术使得该系统可以嵌入到MOOC课程网站。系统使用的主要技术为Javascript技术,研究的主要内容为数据结构和动态演示。
\end{chatext}
\section{Javascript}
\begin{sectext}
Javascript是一种属于网络的脚本语言,已经被广泛用于Web应用开发,常用来为网页添加各式各样交互行为,具有良好的跨平台性,在大多数浏览器的支持下,能够完成许多出色的功能。

在基于MOOC的数据结构动态演示系统中,Javascript主要用来创建数据结构的动态演示,提高网页的交互性。其主要的开发平台有浏览器控制台和node.js平台。浏览器控制台可用于设计、实现过程中的调试,将最终系统的展示在浏览器页面中;而node.js开发平台是一个基于Chrome Javascript运行时平台,有自己的包管理工具,用于方便、快捷的搭建易于扩展的网页应用。
\end{sectext}
\section{数据结构和动态演示}
\begin{sectext}
系统中涉及的数据结构来源于MOOC数据结构课程的教学大纲,同时也参考传统教学的教材中所涉及到的基本数据结构和算法。系统中演示数据结构的参考源代码采用相应教学语言(C语言)进行编写,这些代码与系统中的动画同步演示,可以帮助学习者加深对代码的理解,在实际编程中更好地应用数据结构。演示的内容主要由数据结构和算法构成,其中数据结构不仅仅是数据结构本身的概念,而是通过演示其在算法或实际应用中的初始化、具体操作步骤等以达到课程更加生动有趣的效果。

系统中演示的数据结构及其应用如下:
\begin{itemlist}
\item 栈:算术表达式。
\item 队列:杨辉三角。
\item 矩阵:稀疏矩阵的存储压缩。
\item 树:哈夫曼编码、平衡二叉树。
\item 图:拓扑排序、强连通分量、Dijkstra最短路、Prim最小生成树。
\item 散列表:开地址、链地址解决冲突。
\end{itemlist}

系统中演示的算法如下:
\begin{itemlist}
\item 排序:归并排序、快速排序。
\item 查找:顺序查找、二分查找。
\end{itemlist}

动态演示的实现包括实现其界面绘制、图形绘制和事件的处理,这些则全部由Javascript技术支持。

首先,演示界面相对统一,对多个不同内核的浏览器进行兼容。既可以通过编写简单的HTML和CSS实现,也可以加入相应的Javascript UI库对网页调整。

其次,数据结构图形对于不同的数据结构会有所差异,但一般都是由基本图形构成,即线段、弧线、矩形、圆形等,这些在HTML页面上的绘制可以通过D3.js库完成。D3.js是最流行的可视化库之一,基于数据对HTML文档树进行操作,常用于将数据转化为交互性的图形,支持SVG和Canvas两种模式。

最后,事件的处理,也就是页面交互以及动画过程,需要结合Javascript的设计模式。发布者—订阅者模式,或者是观察者模式,都是处理事件常用的模式。这种模式定义了一种一对多的依赖关系,可以有效的解决对象互相依赖过程中产生的副作用,维护相关对象的一致性。

以上三个模块,由相互独立的Javascript库完成,借助node.js的模块化开发的特点,将模块整合成一个系统的模块,单元间可以共享资源,例如:演示界面的元素等。而每个系统的模块作为一种数据结构的动态演示,可以嵌入到MOOC的课程网站中,也可以独立在单一的网页中显示,从而搭建基于MOOC的数据结构课程动态演示系统。
\end{sectext}
%
