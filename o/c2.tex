% !TEX root = ../o.tex
\chapter{文献综述}
\begin{chatext}
随着互联网的发展以及笔记本电脑和移动设备的普及,数据结构动态演示系统从早期的实验室系统演变为基于网络的在线系统,提供各种基本数据结构的可视化效果。早期的数据结构动画演示系统为单一的客户端软件,即学习者从网络下载软件后在本地电脑中运行。例如:TRAKLA2、JHAVE等。这些系统软件在国内外都可以通过互联网找到相应的介绍网址。而现代的数据结构动画演示系统则多为在线系统,即学习者通过计算机或移动设备中的浏览器软件访问动态演示系统网站进行学习。目前,这些网站大多数由国外开发维护,例如:旧金山大学网站上的数据结构可视化、新加坡国立大学网站上的Visualgo等。
%\footnote{\url{http://www.cs.usfca.edu/~galles/visualization/}}
%\footnote{\url{http://visualgo.net/}}
%\footnote{\url{http://sorting.at/}}
%\footnote{\url{http://theory.stanford.edu/~amitp/GameProgramming/}}

数据结构课程动态演示系统的设计要素有三点:数据结构和算法的分类、动态演示的交互性以及系统的兼容性。

对于数据结构和算法的分类,系统可分为单一可视化类型和统一完备类型。单一可视化指的是系统只为某一个特定的或者某一类数据结构、算法提供动态演示。常见单一动态演示包括数列排序、图的数据结构及其算法等。例如,在网站上介绍了各种排序算法,包括:插入排序、冒泡排序、快速排序等。又例如,Amit的Astar介绍网站用动态演示的方式解释了图的数据结构及其算法和应用,包括BFS、Dijkstra、Astar等。而统一完备则指的是系统地对常见或非常见的数据结构和算法进行动态演示,例如:Visualgo网站用于数据结构与算法课程的教学等。

对于动态演示的交互性,则一共有六个级别,对应于数据结构课程的动态演示系统的六个级别如下:
\begin{itemlist}
\item 无交互性,只能单一地播放数据结构的动态演示。
\item 能够控制动态演示的速度。
\item 对于动态演示的数据结构有反馈信息,例如:提问、测试等。
\item 能够输入数据,得到不同的动态演示效果。
\item 能够编写自己的数据结构,并进行动态演示。
\item 尽可能多地收集用户反馈,并提高用户体验。
\end{itemlist}

这六个级别从低到高能够逐步完善用户体验,加强学习效果。

对于系统的兼容性,如本章节开头提到的内容,越来越多的系统已经移植到了互联网上。但是这些网站使用的技术各所不同,包括:Java、Flash和Javascript等。其中用Java的小应用程序和Flash程序都需要浏览器加载插件才可以运行,而使用Javascript时通常也会使用HTML5、Canvas和SVG等技术,这些都需要版本更高版本的浏览器支持,例如:Internet Explorer 10、高版本号的Firefox或者Google Chrome等。
\end{chatext}
