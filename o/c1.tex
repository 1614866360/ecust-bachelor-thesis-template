% !TEX root = ../o.tex
\chapter{研究背景}
% \ifthenelse{\equal{\thesistype}{}}{正文}{不是正文}

\section{背景介绍}
\begin{sectext}
MOOC(Massive Open Online Course)指的是大规模开放在现课堂。2012年起,许多知名大学陆续登录到在线学习网络平台,在网上提供免费课程。现在,著名的在网络平台包括Coursera、Udacity和edX等。随着课程提供商的兴起,世界各地的学生拥有了更多系统学习的机会。这些平台和真正的大学一样,提供的课程大部分针对高等教育,有一套自己的学习和管理系统。

由于计算机和互联网技术的迅速发展,计算机相关的课程在MOOC上也都变得非常热门,其中也包括数据结构这门课程。数据结构课程作为计算机科学中一门综合性较强的专业基础课程,研究的内容包括非数值计算程序设计问题中计算机操作的对象以及它们之间的关系和运算等。而这部分内容往往都是抽象的,初学者通常难以对课程中的概念有比较直观的理解。

本研究通过结合MOOC课程所具有的特征,设计并实现一个数据结构课程动态演示系统。
\end{sectext}
\section{研究意义}
\begin{sectext}
数据结构课程动态演示能够帮助学习者更好地理解抽象的数据结构概念和复杂的算法步骤。系统通过把抽象的数据结构复杂的算法转换为形象生动的动画过程,清晰地向学习者展现数据结构的操作步骤,在演示动画中可以关注每一个执行细节,这样更有利于课后自己动手编写数据结构和算法相关的程序。同时,学习者也不会对于数据结构或算法的概念亦或是编程感到枯燥无聊,更加积极得参与课程的学习中。

本研究基于MOOC的特点对数据结构课程动态演示系统进行设计并实现,进一步补充了MOOC课程中所存在的不足之处。MOOC课程是将传统课程讲授与在线网络相结合,一方面发挥了网络优势,另一方面也带来了平台自身的局限性,其特点集中表现在五个方面,具体如下:
\begin{itemlist}
\item 大多数的在线课程均有自己的上课周期,大约为1--3个月。
\item 上课内容碎片化,主要以知识点进行划分。
\item 多数课程为教师引导型授课。因为缺乏师生互动性,要想到达理想授课效果,需要把讲授知识简化、串引,并附有生动易于理解的动画。
\item 授课除包含教师讲解外,还会有相应的作业和考试机制。每门课程都有相应的教师与学生的讨论.教师需要在授课以外时间完成学生的各种提问。
\item 与传统课堂教学相比MOOC课程的学生来自全球各地语言的互通成为了MOOC课程翻译的重要一步。
\end{itemlist}

针对以上五点,数据结构课程动态演示系统的设计与实现将帮助参与到MOOC数据结构课程的学习者更有效的学习。与以往在课程的课件中添加参考资料或者外部访问的链接不同,将系统嵌入到MOOC的课程中可达到以下三个效果:
\begin{itemlist}
\item 知识点集中,分类目录清晰,帮助学习者全面掌握并巩固课程基本概念。
\item 交互式动态演示和图形化界面可以提高学习者的兴趣,消除学习者的语言障碍,从而更有效地参与到在线的课堂和课后的讨论中。
\item 对于基本的习题答疑,可以直接参考动态演示内容,有助于提高学习效率。
\end{itemlist}
\end{sectext}
